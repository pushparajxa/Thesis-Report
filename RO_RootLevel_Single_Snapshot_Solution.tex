  %%%%%%%%%%%%%%%%%%%%%%%%%%%%%%%%%%%%%%% -*- coding: utf-8; mode: latex -*- %%
  %
%%%%%                         CHAPTER
 %%%
  %

% $Id: 1020-lorem-ipsum.tex,v 1.2 2009/06/19 15:51:46 david Exp $
% $Log: 1020-lorem-ipsum.tex,v $
% Revision 1.2  2009/06/19 15:51:46  david
% *** empty log message ***
%
% Revision 1.1  2007/11/23 09:52:39  david
% *** empty log message ***
%
%

  %%%%%%%%%%%%%%%%%%%%%%%%%%%%%%%%%%%%%%%%%%%%%%%%%%%%%%%%%%%%%%%%%%%%%%%%%%%%%
  %
%%%%%                           HEAD MATTER
 %%%
  %

\chapter{Read-Only Root Level Single Snapshot}
%\addcontentsline{lof}{chapter}{\thechapter\quad Lorem Ipsum}
%\addcontentsline{lot}{chapter}{\thechapter\quad Lorem Ipsum}
\label{ch:RORLSSSolution}

Following conditions are applied to the solution
\begin{enumerate}
\item Creation of directories with Quota, either name-space quota or disk-space quota is not allowed.
\item Each file consists of blocks. Each blocks-size is typically 64 MB but can be set to any value.Blocks should be written completely.
\end{enumerate}


  %%%%%%%%%%%%%%%%%%%%%%%%%%%%%%%%%%%%%%%%%%%%%%%%%%%%%%%%%%%%%%%%%%%%%%%%%%%%%
  %
%%%%%                        FIRST SECTION
 %%%
  %

\section{Modifications to the Schema}

Following columns need to be added to the Inodes table described in the schema \ref{fig:HDFS_table_schema} of HOP File System.
\begin{enumerate}
\item isDeleted \\\\
\begin{tabular}{|c|p{15cm}|}
\hline
Value&Summary\\
\hline
0&Indicates that this Inode is not deleted.\\
\hline
1&Indicates that this Inode deleted after Root Level snapshot was taken.\\ 
\hline
\end{tabular} \\
\item status \\\\
\begin{tabular}{|c|p{15cm}|}
\hline
Value&Summary\\
\hline
0&Indicates that this Inode was created before taking Root Level Snapshot.\\
\hline
2&Indicates that this Inode created before taking Root Level snapshot but modified after that.\\ 
\hline
3&Indicates that this Inode was created after taking Root Level snapshot.\\ 
\hline
\end{tabular} \\

\end{enumerate}
\pagebreak

Following Columns should be added to BlockInfos table described in the schema \ref{fig:HDFS_table_schema}.
\begin{enumerate}
\item status \\\\
\begin{tabular}{|c|p{15cm}|}
\hline
Value&Summary\\
\hline
0&Indicates that this Block was created before taking Root Level Snapshot.\\
\hline
2&Indicates that this Block created before taking Root Level snapshot but modified after that.\\ 
\hline
3&Indicates that this Block was created after taking Root Level snapshot.\\ 
\hline
\end{tabular} \\
\end{enumerate}

\section{Rules for Modifying the fileSystem meta-data}
Following rules apply when client issues operations described on \ref{NameNode_Ops} after root level snapshot had been taken.\\
HOP-HDFS as well as Apache HDFS allow only appends at the end of file.Both allow overwriting of an existing file.
\begin{enumerate}
\item If an inode(file or directory) is created after taking root level snapshot, its status is set to 3.
\item If an inode row is modified and its status is 0, then, a back-up of current row is saved with id = -(current id), parent\_ id=-(current parent\_ id)[To prevent sql query retrieving the back-up rows while 'ls' command issued, parent id is set to negative of original].The status of current row is changed to 2.
\item If a block is created after taking root level snapshot,its status is set to 3.
\item If a block is modified by appending data to it and its status is 0, then, a back-up of current row is saved with block\_ id = -(current block\_ id) and inode\_ id = -(current inode\_ id)[since two block info rows can't have same block index id when retrieved with a parent id].The status of current row is changed to 2.
\item Deletion of a directory or file after root level snapshot was taken.
Children of the INode to be deleted are examined in depth-first manner.
\pagebreak
\begin{verbatim}
void deleteWithSnapshotAtRootTaken(INode targetNode){

        Stack<INode> stck = new Stack<INode>();
        INode tempNode;
        List<INode> children;
        INode[] inodesTemp;
        INode removedInode;

        stck.add(targetNode);

        while (!stck.empty()) {
            tempNode = stck.pop();
            tempSts = tempNode.getStatus();
            tempStr = tempNode.getFullPathName();
           
            if (tempNode.getIsDeleted() == 1) {
                //This node is already marked deleted, so nothing to do.
                continue;
            }
            /*
             * This Inode can be a directory or file also it can be new or modified or original.
             */
            if (tempNode instanceof INodeDirectory) {
                children = ((INodeDirectory) tempNode).getChildren();
                if (children != null && !children.isEmpty()) {
                    stck.push(tempNode);
                    for (INode n : children) {
                        stck.push(n);
                    }
                } else {
                    if(tempNode.getStatus()==SnapShotConstants.New){
                        //delete completely this directory Inode
                        EntityManager.remove(tempNode);
                    }
                    else{
                    //Set isDeleted = 1.
                        tempNode.setIsDeletedNoPersistance(SnapShotConstants.isDeleted);
                    }
                }
            } else if (tempNode instanceof INodeFile || tempNode instanceof INodeSymlink) {
                if (tempSts == SnapShotConstants.New) {
                    //We can delete this file permanently and update the ansectors about changes in the quota.
                    //Remove the blocks associated with this file permanently.
                    }
                } else if (tempSts == SnapShotConstants.Original || tempSts == SnapShotConstants.Modified) {
                    //Set isDeleted = 1.
                    tempNode.setIsDeletedNoPersistance(1);
                }
            } 
        }// End of while loop
     }//End of method      
            
\end{verbatim}
\item Renaming/Moving an INode(File or Directory)
\begin{verbatim}
void renameINode(INode src, INode dst){
        1. Update the modification time of parent of src.
        2. Update the modification time of parent of dst.
        3. deletWithSnapshotAtRootTaken(dst).
        4. Change the parent\_ id of src to dst.parent\_ id.
}
\end{verbatim}

\end{enumerate}


  %%%%%%%%%%%%%%%%%%%%%%%%%%%%%%%%%%%%%%%%%%%%%%%%%%%%%%%%%%%%%%%%%%%%%%%%%%%%%
  %
%%%%%                      SECOND SECTION
 %%%
  %

\section{Roll Back}
 Following algorithm is used to roll back the file-system to the state at the time when Root Level Snapshot was taken.\\\\
For INodes:
\begin{enumerate}
\item Delete from INodes where status=2 or status=3
\item Update INodes set isDeleted=0 where id$>$0 and isDeleted=1
\item Update INodes set id = -id, parent\_ id = -parent\_ id where id$<$0
\end{enumerate}
For Blocks:
\begin{enumerate}
\item Delete from Block\_ Info where status=2 or status=3
\item Update Block\_ Info set block\_ id = -block\_ id, inode\_ id = -inode\_ id where id$<$0
\item Delete from Block\_ Info where block\_ id$<$0
\end{enumerate}
\label{alg:ROSSRB}


  %%%%%%%%%%%%%%%%%%%%%%%%%%%%%%%%%%%%%%%%%%%%%%%%%%%%%%%%%%%%%%%%%%%%%%%%%%%%%
  %
%%%%%                         ANOTHER SECTION
 %%%
  %


  %
 %%%
%%%%%                        THE END
  %
  %%%%%%%%%%%%%%%%%%%%%%%%%%%%%%%%%%%%%%%%%%%%%%%%%%%%%%%%%%%%%%%%%%%%%%%%%%%%%

%%% Local Variables: 
%%% mode: latex
%%% TeX-master: "tese"
%%% End: 
