%--------------------------------------------
%
% Package pgfplotstable
%
% Provides support to read and work with abstact numeric tables of the
% form
%
% COLUMN1	COLUMN2 COLUMN3
% 1 		2		3
% 4			4		552
% 1e124		0.00001	1.2345e-12
% ...
%
% Copyright 2007-2010 by Christian Feuersänger.
%
% This program is free software: you can redistribute it and/or modify
% it under the terms of the GNU General Public License as published by
% the Free Software Foundation, either version 3 of the License, or
% (at your option) any later version.
% 
% This program is distributed in the hope that it will be useful,
% but WITHOUT ANY WARRANTY; without even the implied warranty of
% MERCHANTABILITY or FITNESS FOR A PARTICULAR PURPOSE.  See the
% GNU General Public License for more details.
% 
% You should have received a copy of the GNU General Public License
% along with this program.  If not, see <http://www.gnu.org/licenses/>.
%
%--------------------------------------------


% This file provides a high-level table manipulation and typesetting
% package.
%
% see pgfplotstableshared.code.tex for the low level routines.


\newif\ifpgfplotstabletypesetdebug
\newif\ifpgfplotstable@sort
\newif\ifpgfplotstabletypesetskipcoltypes
\newif\ifpgfplotstabletypesetresult
\newif\ifpgfplotstableuserow
\newif\ifpgfplotstabletypeset@includeoutfiles
\newif\ifpgfplotstabletypeset@force@remake
\newif\ifpgfplotstable@disable@rowcolstyles

\let\pgfplotstable@outfile=\w@pgf@writea

\input pgfplotstable.coltype.code.tex

% #1= floating point number
% #2= TeX code to execute if #1 == 0
% #3= TeX code to execute if #1 != 0
\def\pgfplots@ifzero#1#2#3{%
	\expandafter\pgfmathfloat@decompose@F#1\relax\pgfmathfloat@a@S
	\ifnum\pgfmathfloat@a@S=0 #2\else#3\fi
}%

% just make sure no-one complaints if manual example are used without
% booktabs loaded:
\pgfutil@IfUndefined{toprule}{%
	\def\toprule{\pgfplots@assert@LaTeX@package@loaded{booktabs}{\string\toprule}}%
	\def\midrule{\pgfplots@assert@LaTeX@package@loaded{booktabs}{\string\midrule}}%
	\def\bottomrule{\pgfplots@assert@LaTeX@package@loaded{booktabs}{\string\bottomrule}}%
}{\relax}%

% keys which are NOT predefined:
% /pgfplots/table/alias/<col alias>/.initial={<real col>}
% /pgfplots/table/columns/<col name>/.style={}
% /pgfplots/table/display columns/<col index>/.style={}
% /pgfplots/table/create on use/<col name>/.style={create options}
\pgfkeys{%
	/pgfplots/table/disable rowcol styles/.is if=pgfplotstable@disable@rowcolstyles,
	/pgfplots/table/disable rowcol styles/.default=true,
	/pgfplots/table/row predicate/.code={},
	/pgfplots/table/skip rows between index/.style 2 args={%
		/pgfplots/table/row predicate/.append code={%
			\ifnum##1<#1\relax
			\else
				\ifnum##1<#2\relax
					\pgfplotstableuserowfalse
				\fi
			\fi}
	},
	/pgfplots/table/select equal part entry of/.style 2 args={%
		/pgfplots/table/row predicate/.code={%
			\pgfplotstableuserowtrue
			\begingroup
			% this group re-uses counters as temporary variables.
			\c@pgfplotstable@colindex=\pgfplotstablerows\relax
			\divide\c@pgfplotstable@colindex by#2\relax
			\edef\pgfplotstablepartsize{\the\c@pgfplotstable@colindex}%
			% This here should create empty cells such that
			% remaining entries are distributed equally:
			\c@pgfplotstable@rowindex=\c@pgfplotstable@colindex
			\multiply\c@pgfplotstable@rowindex by#2\relax
			\ifnum\c@pgfplotstable@rowindex<\pgfplotstablerows\relax
				\c@pgfplotstable@colindex=\pgfplotstablerows\relax
				\advance\c@pgfplotstable@colindex by-\c@pgfplotstable@rowindex
				\advance\c@pgfplotstable@colindex by\pgfplotstablepartsize				
				\edef\pgfplotstablepartsize{\the\c@pgfplotstable@colindex}%
			\fi
			%
			\multiply\c@pgfplotstable@colindex by#1\relax
			\ifnum##1<\c@pgfplotstable@colindex\relax
				\aftergroup\pgfplotstableuserowfalse
			\else
				\advance\c@pgfplotstable@colindex by\pgfplotstablepartsize\relax
				\ifnum##1<\c@pgfplotstable@colindex\relax
				\else
					\aftergroup\pgfplotstableuserowfalse
				\fi
			\fi
			\endgroup
		}%
	},
	% #1: colname
	/pgfplots/table/unique/.code={%
		\pgfkeys{/pgfplots/table/row predicate/.append code={%
			\ifnum\pgfplotstablerow=0
				\ifnum\pgfplotstablecol=0
					\pgfutil@ifundefined{pgfplotstable@unique@bitlist@backup}{%
					}{%
						\pgfplots@error{It seems there are *multiple* 'unique' keys running on one table. That doesn't work correctly, I guess. Consider using '\string\pgfplotstableset{row predicate/.code={}}' to reset it.}%
					}%
					% PREPARE ONCE! Assemble the boolean results into
					% a list which is used for the complete table:
					\def\pgfplotstable@loc@TMPd{\pgfplotstablegetcolumnfromstruct{#1}\of}%
					\expandafter\pgfplotstable@loc@TMPd\pgfplotstablename\to\pgfplotstable@unique@col
					\pgfplotslistnewempty\pgfplotstable@unique@bitlist
					\let\pgfplotstable@unique@LAST=\pgfutil@empty
					\pgfplotslistforeachungrouped\pgfplotstable@unique@col\as\pgfplotstable@unique@cur{%
						\def\pgfplotstable@unique@bit{1}%
						\ifx\pgfplotstable@unique@LAST\pgfutil@empty
						\else
							\ifx\pgfplotstable@unique@LAST\pgfplotstable@unique@cur
								\def\pgfplotstable@unique@bit{0}%
							\fi
						\fi
						\ifpgfplotstabletypesetdebug
							\pgfplots@message{unique={#1}: cur == last <=> ( \pgfplotstable@unique@cur\space == \pgfplotstable@unique@LAST ) = \pgfplotstable@unique@bit.}%
						\fi
						\expandafter\pgfplotslistpushback\pgfplotstable@unique@bit\to\pgfplotstable@unique@bitlist
						\let\pgfplotstable@unique@LAST=\pgfplotstable@unique@cur
					}%
					\global\let\pgfplotstable@unique@bitlist@backup=\pgfplotstable@unique@bitlist
				\else
					\pgfutil@ifundefined{pgfplotstable@unique@bitlist@backup}{%
						% this sanity checking is NOT fool proof: it
						% fails if there are different occurances of
						% unique in the same file
						\pgfplotsthrow{invalid argument}{Sorry, the row predicate /pgfplots/table/unique={#1} has been used in the wrong context: it needs to be invoked for the very first processed column, not column no \pgfplotstablecol. Please provide it as argument to \string\pgfplotstabletypeset[unique={#1}] and not inside of column-specific styles}\pgfeov%
					}{}%
				\fi
				% acquire the assembled list here: we'll do a lot of
				% popfronts with it.
				\let\pgfplotstable@unique@bitlist=\pgfplotstable@unique@bitlist@backup
			\fi
			\pgfplotslistcheckempty\pgfplotstable@unique@bitlist
			\ifpgfplotslistempty
				% should not happen!
				\pgfplotstableuserowfalse
			\else
				\pgfplotslistpopfront\pgfplotstable@unique@bitlist\to\pgfplots@loc@TMPa
				\if\pgfplots@loc@TMPa0%
					\pgfplotstableuserowfalse
				\fi
			\fi
			% cleanup:
			{%
				\count0=\pgfplotstablecol\relax \advance\count0 by1
				\ifnum\count0=\pgfplotstablecols\relax
					\count0=\pgfplotstablerow\relax \advance\count0 by1
					\ifnum\count0=\pgfplotstablerows\relax
						\global\let\pgfplotstable@unique@bitlist@backup=\relax
					\fi
				\fi
			}%
		}}%
	},%
	% columns={name1,name2}
	% or
	% columns={[index]2,name2,name3,[index]5}
	/pgfplots/table/columns/.initial=,
	%
	% this choice allows 
	% \pgfplotstableset{
	%    column name={}, % means: the column's display name is an empty string!
	%    column name=\pgfkeysnovalue, % means: no value specified. In this case, 
	%             the column's display name will default to column's name.
	% }
	/pgfplots/table/column name/.initial=\pgfkeysnovalue,
	%
	% this thing here allows to MODIFY 'column name'.
	%
	% Argument #1 is the current column name, that means after
	% evaluating 'column name'. If this key changes anything, it
	% should write its result back into 'column name'.
	%
	% That means you can use 'column name' to assign the name as such
	% and 'assign column name' to generate final TeX code (for example
	% to insert \multicolumn{1}{c}{#1} or so).
	% default is empty which means no change.
	%/pgfplots/table/assign column name/.code={
	%	\pgfkeyssetvalue{/pgfplots/table/column name}{#1}%
	%},
	%
	%
	%
	% A style which inserts \multicolumn{1}{#1}{<column name>} for
	% each column name.
	% The column name as such can be set with the 'column name' option.
	/pgfplots/table/multicolumn names/.style={%
		/pgfplots/table/assign column name/.code={%
			\pgfkeyssetvalue{/pgfplots/table/column name}{\multicolumn{1}{#1}{##1}}%
		}%
	},
	/pgfplots/table/multicolumn names/.default=c,
	/pgfplots/table/dec sep align/.code={%
		\pgfplots@assert@LaTeX@package@loaded{array}{dec sep align}%
		\def\pgfplotstable@scisepalign@headeralign{#1}%
		\pgfkeysalso{%
			/pgf/number format/assume math mode,
			/pgf/number format/@dec sep mark={$&$},
			/pgfplots/table/assign column name/.code={%
				\pgfkeyssetvalue{/pgfplots/table/column name}{\multicolumn{2}{#1}{##1}}%
			},%
			/pgfplots/table/column type={%
				r<{\pgfplotstableresetcolortbloverhangright}%
				@{}%
				l<{\pgfplotstableresetcolortbloverhangleft}%
			},
			/pgfplots/table/assign cell content/.code={%
				\def\pgfmathresult{##1}%
				\let\continue=\pgfutil@empty
				% allow special handling:
				\pgfplots@invoke@pgfkeyscode{/pgfplots/table/dec sep align/process/.@cmd}{##1}%
				\ifx\continue\pgfutil@empty
					% nothing has changed. Processed as usual:
					\ifx\pgfmathresult\pgfutil@empty
						\def\pgfmathresult{&}%
					\else
						% -6.90000001e-01 -> \meaning\pgfmathresult = macro:->-0$&$.69
						\pgfmathprintnumberto{\pgfmathresult}\pgfmathresult%
						% now make sure we have math mode for the single
						% columns:
						\expandafter\def\expandafter\pgfmathresult\expandafter{\expandafter$\pgfmathresult$}%
					\fi
				\fi
				\pgfkeyslet{/pgfplots/table/@cell content}\pgfmathresult
			},
		}%
	},
	/pgfplots/table/dec sep align/.default=c,
	%
	% A part of dec sep align which can be used to process special
	% cases.
	%
	% #1: the unprocessed input argument.
	% PRECONDITION:
	%   \pgfmathresult contains '#1', not more.
	%   \continue is empty.
	%
	% POSTCONDITION:
	% 	If \continue is empty, `dec sep align' will continue just as
	% 	if the 'process' key hadn't been invoked.
	% 	It will, however, use the current value of \pgfmathresult.
	%
	% 	If \continue is NOT empty, for example \def\continue{0}, 
	% 	`dec sep align' assumes that \pgfmathresult contains the
	% 	completely typeset cell, including any alignment material.
	/pgfplots/table/dec sep align/process/.code=,%
	/pgfplots/table/dec sep align/no unbounded/.style={%
		% FIXME : this thing doesn't work as intended! It looks ugly!
		/pgfplots/table/dec sep align/process/.code={%
			\ifx\pgfmathresult\pgfutil@empty
			\else
				\pgfmathfloatparsenumber\pgfmathresult
				\pgfmathfloatiffinite{\pgfmathresult}{%
					% do nothing, just communicate the parsed
					% \pgfmathresult
				}{%
					\begingroup
						\t@pgfplots@toka=\expandafter{\pgfplotstable@scisepalign@headeralign}%
						\t@pgfplots@tokb={\pgfkeyslet{/pgf/number format/@dec sep mark}\pgfutil@empty\pgfmathprintnumber}%
						\t@pgfplots@tokc=\expandafter{\pgfmathresult}%
						\xdef\pgfplotstable@glob@TMPc{%
							\noexpand\multicolumn{2}{\the\t@pgfplots@toka}%
								{\the\t@pgfplots@tokb{\the\t@pgfplots@tokc}}%
						}%
					\endgroup
					\let\pgfmathresult=\pgfplotstable@glob@TMPc
					\def\continue{0}%
				}%
			\fi
		},%
	},%
	/pgfplots/table/sci sep align/.code={%
		\pgfplots@assert@LaTeX@package@loaded{array}{sci sep align}%
			\pgfkeysalso{%
			/pgf/number format/assume math mode,
			/pgf/number format/@sci exponent mark={$&$},
			/pgfplots/table/assign column name/.code={%
				\pgfkeyssetvalue{/pgfplots/table/column name}{\multicolumn{2}{#1}{##1}}%
			},%
			/pgfplots/table/column type={%
				r<{\pgfplotstableresetcolortbloverhangright}%
				@{}%
				l<{\pgfplotstableresetcolortbloverhangleft}%
			},
			/pgfplots/table/assign cell content/.code={%
				\def\pgfmathresult{##1}%
				\ifx\pgfmathresult\pgfutil@empty
					\def\pgfmathresult{&}%
				\else
					\pgfmathprintnumberto{##1}\pgfmathresult%
					\expandafter\pgfutil@in@\expandafter&\expandafter{\pgfmathresult}%
					% now make sure we have math mode for the single
					% columns:
					\ifpgfutil@in@
						\expandafter\def\expandafter\pgfmathresult\expandafter{\expandafter$\pgfmathresult$}%
					\else
						\expandafter\def\expandafter\pgfmathresult\expandafter{\expandafter\pgfutilensuremath\expandafter{\pgfmathresult}&}%
					\fi
				\fi
				\pgfkeyslet{/pgfplots/table/@cell content}\pgfmathresult
			},
		}%
	},
	/pgfplots/table/sci sep align/.default=c,
	%
	% A style which can be used together with the 'dcolumn' package by
	% David Carlisle.
	% #1: the dcolumn type, defaults to 'D{.}{.}{2}'
	% #2: the column name type, defaults to 'c'
	/pgfplots/table/dcolumn/.style 2 args={%
		/pgf/number format/assume math mode,
		column type={#1},
		multicolumn names=#2,
	},
	/pgfplots/table/dcolumn/.default={D{.}{.}{2}}{c},
	/pgfplots/table/column type/.initial={c},
	/pgfplots/table/every even row/.style={},
	/pgfplots/table/every odd row/.style={},
	/pgfplots/table/every last row/.style={},
	/pgfplots/table/every first row/.style={},
	/pgfplots/table/every head row/.style={},
	/pgfplots/table/every first column/.style={},
	/pgfplots/table/every last column/.style={},
	/pgfplots/table/every even column/.style={},
	/pgfplots/table/every odd column/.style={},
	/pgfplots/table/every nth row/.code 2 args={%
		\pgfplotstabletypeset@append@every@nth@row{#1}{#2}%
	},
	/pgfplots/table/every nth row/.style 2 args/.code 2 args={%
		\pgfplotstabletypeset@append@every@nth@row{#1}{#2}%
	},
	/pgfplots/table/before row/.initial=,
	/pgfplots/table/after row/.initial=,
	/pgfplots/table/begin table/.initial={\begin{tabular}},
	/pgfplots/table/end table/.initial={\end{tabular}},
	/pgfplots/table/outfile/.initial=,
	/pgfplots/table/include outfiles/.is if=pgfplotstabletypeset@includeoutfiles,
	/pgfplots/table/include outfiles/.default=true,
	/pgfplots/table/force remake/.is if=pgfplotstabletypeset@force@remake,
	/pgfplots/table/force remake/.default=true,
	/pgfplots/table/write to macro/.initial=,
	/pgfplots/table/typeset/.is if=pgfplotstabletypesetresult,
	/pgfplots/table/typeset=true,
	/pgfplots/table/skip coltypes/.is if=pgfplotstabletypesetskipcoltypes,
	/pgfplots/table/skip coltypes/.default=true,
	/pgfplots/table/debug/.is if=pgfplotstabletypesetdebug,
	/pgfplots/table/debug level/.initial=0,%
	%
	% will be redefined by |assign cell content| for every cell:
	/pgfplots/table/@cell content/.initial=,
	%
	% #1: the cells content as it has been found in the input table
	% this command key should somehow fill |cell content|.
	/pgfplots/table/assign cell content/.code={%
		\def\pgfmathresult{#1}%
		\ifx\pgfmathresult\pgfutil@empty
		\else
			\pgfmathprintnumberto{#1}\pgfmathresult%
		\fi
		\pgfkeyslet{/pgfplots/table/@cell content}\pgfmathresult
	},
	%
	% this here is the default formatting. It uses
	% \pgfmathprintnumber.
	/pgfplots/table/assign cell content as number/.code={%
		\def\pgfmathresult{#1}%
		\ifx\pgfmathresult\pgfutil@empty
		\else
			\pgfmathprintnumberto{#1}\pgfmathresult%
		\fi
		\pgfkeyslet{/pgfplots/table/@cell content}\pgfmathresult
	},
	/pgfplots/table/numeric type/.code={%
		\pgfkeysgetvalue{/pgfplots/table/assign cell content as number/.@cmd}\pgfplotstable@loc@TMPa%
		\pgfkeyslet{/pgfplots/table/assign cell content/.@cmd}\pgfplotstable@loc@TMPa
	},
	/pgfplots/table/string type/.style={%
		/pgfplots/table/assign cell content/.style={%
			/pgfplots/table/@cell content={##1}%
		}%
	},%
	/pgfplots/table/verb string type/.style={%
		/pgfplots/table/assign cell content/.code={%
			\def\pgfplotstable@loc@TMPa{##1}%
			\pgfplots@command@to@string\pgfplotstable@loc@TMPa\pgfplotstable@loc@TMPa
			\pgfkeyslet{/pgfplots/table/@cell content}{\pgfplotstable@loc@TMPa}%
		}%
	},%
	/pgfplots/table/numeric as string type/.style={%
		/pgfplots/table/assign cell content/.code={%
			\def\pgfmathresult{##1}%
			\ifx\pgfmathresult\pgfutil@empty
			\else
				\pgfmathifisint{##1}{\let\pgfmathresult=\pgfretval}{\pgfmathfloattosci{\pgfretval}}%
			\fi
			\pgfkeyslet{/pgfplots/table/@cell content}\pgfmathresult
		}%
	},%
	/pgfplots/table/date type/.style={%
		/pgfplots/table/assign cell content/.code={%
			\begingroup
			\pgfcalendardatetojulian{##1}\c@pgfplotstable@counta
			\pgfcalendarjuliantodate{\c@pgfplotstable@counta}\year\month\day
			\pgfcalendarjuliantoweekday\c@pgfplotstable@counta\c@pgf@countc
			\edef\weekday{\the\c@pgf@countc }%
			\edef\weekdayname{\pgfcalendarweekdayname\c@pgf@countc}%
			\edef\weekdayshortname{\pgfcalendarweekdayshortname\c@pgf@countc}%
			\edef\monthname{\pgfcalendarmonthname\month}%
			\edef\monthshortname{\pgfcalendarmonthshortname\month}%
			\xdef\pgfplots@glob@TMPa{#1}%
			\endgroup
			\pgfkeyslet{/pgfplots/table/@cell content}\pgfplots@glob@TMPa%
		}%
	},%
	/pgfplots/table/date type/.default={\year/\month/\day},%
	/pgfplots/table/set content/.style={%
		/pgfplots/table/postproc cell content/.style={%
			/pgfplots/table/@cell content={#1}%
		}%
	},%
	%
	/pgfplots/table/postproc cell content/.code={},
	/pgfplots/table/preproc cell content/.code={},
	%
	/pgfplots/table/clear infinite/.style={%
		/pgfplots/table/preproc cell content/.append code={%
			\pgfkeysgetvalue{/pgfplots/table/@cell content}\pgfmathresult
			\ifx\pgfmathresult\pgfutil@empty
			\else
				\pgfmathfloatparsenumber{\pgfmathresult}%
				\let\pgfmatharga=\pgfmathresult
				{%
				\pgfmathfloatgetflags\pgfmatharga\c@pgfplotstable@counta
				\xdef\pgfplots@glob@TMPc{\the\c@pgfplotstable@counta}%
				}%
				\ifnum\pgfplots@glob@TMPc<3
					\pgfmathfloattosci@\pgfmathresult
					\pgfkeyslet{/pgfplots/table/@cell content}{\pgfmathresult}%
				\else
					\pgfkeyslet{/pgfplots/table/@cell content}{\pgfutil@empty}%
				\fi
			\fi
		}
	},
	/pgfplots/table/string replace/.style 2 args={%
		/pgfplots/table/preproc cell content/.append code={%
			\pgfkeysgetvalue{/pgfplots/table/@cell content}\pgfmathresult
			\def\pgfplots@loc@TMPa{#1}%
			\ifx\pgfmathresult\pgfplots@loc@TMPa
				\def\pgfplots@loc@TMPb{#2}%
				\pgfkeyslet{/pgfplots/table/@cell content}{\pgfplots@loc@TMPb}%
			\fi
		}
	},
	/pgfplots/table/string replace*/.style 2 args={%
		/pgfplots/table/preproc cell content/.append code={%
			\pgfkeysgetvalue{/pgfplots/table/@cell content}\pgfmathresult
			\def\pgfplots@loc@TMPa{\pgfplotsutilstrreplace{#1}{#2}}%
			\expandafter\pgfplots@loc@TMPa\expandafter{\pgfmathresult}%
			\pgfkeyslet{/pgfplots/table/@cell content}{\pgfplotsretval}%
		}
	},
	/pgfplots/table/preproc/expr/.code={%
		\ifpgfplots@usefpu
			\pgfkeysalso{/pgf/fpu=true,/pgf/fpu/output format=sci}%
		\fi
		\expandafter\def\csname pgfplotstable@preproc@expr@thisrow@\pgfplotstablecolname\endcsname{\pgfkeysvalueof{/pgfplots/table/@cell content}}%
		\def\thisrow##1{%
			\pgfutil@ifundefined{pgfplotstable@preproc@expr@thisrow@##1}{%
				--inaccessable--%
			}{%
				\csname pgfplotstable@preproc@expr@thisrow@##1\endcsname
			}%
		}%
		\pgfkeysalso{/pgfplots/table/preproc cell content/.append code={%
				\pgfkeysgetvalue{/pgfplots/table/@cell content}\pgfmathresult
				\ifx\pgfmathresult\pgfutil@empty
				\else
					\pgfmathparse{#1}%
					\pgfkeyslet{/pgfplots/table/@cell content}\pgfmathresult%
				\fi
			}%
		}%
	},
	/pgfplots/table/multiply -1/.style={%
		/pgfplots/table/preproc cell content/.append code={%
			\pgfkeysgetvalue{/pgfplots/table/@cell content}\pgfmathresult
			\ifx\pgfmathresult\pgfutil@empty
			\else
				\pgfmathfloatparsenumber{\pgfmathresult}%
				\let\pgfmatharga=\pgfmathresult
				\pgfmathfloatcreate{2}{1.0}{0}%
				\let\pgfmathargb=\pgfmathresult
				\pgfmathfloatmultiply@{\pgfmatharga}{\pgfmathargb}%
				\pgfmathfloattosci@\pgfmathresult
				\pgfkeyslet{/pgfplots/table/@cell content}{\pgfmathresult}%
			\fi
		}
	},
	/pgfplots/table/multiply with/.style={/pgfplots/table/multiply by={#1}},%
	/pgfplots/table/multiply by/.code={%
		\pgfmathfloatparsenumber{#1}%
		\let\pgfplotstable@scale=\pgfmathresult
		\pgfkeysalso{
			/pgfplots/table/preproc cell content/.append code={%
				\pgfkeysgetvalue{/pgfplots/table/@cell content}\pgfmathresult
				\ifx\pgfmathresult\pgfutil@empty
				\else
					\pgfmathfloatparsenumber{\pgfmathresult}%
					\let\pgfmatharga=\pgfmathresult
					\pgfmathfloatmultiply@{\pgfmatharga}{\pgfplotstable@scale}%
					\pgfmathfloattosci@\pgfmathresult
					\pgfkeyslet{/pgfplots/table/@cell content}{\pgfmathresult}%
				\fi
			}%
		}%
	},
	/pgfplots/table/divide by/.code={%
		\pgfkeysalso{/pgfplots/table/multiply by=#1}%
		\let\pgfplotstable@divisor=\pgfplotstable@scale
		\pgfmathfloatcreate{1}{1.0}{0}%
		\let\pgfplotstable@ONE=\pgfmathresult
		\pgfmathfloatdivide@{\pgfplotstable@ONE}{\pgfplotstable@divisor}%
		\let\pgfplotstable@scale=\pgfmathresult
	},
	/pgfplots/table/sqrt/.style={%
		/pgfplots/table/preproc cell content/.append code={%
			\pgfkeysgetvalue{/pgfplots/table/@cell content}\pgfmathresult
			\ifx\pgfmathresult\pgfutil@empty
			\else
				\pgfmathfloatparsenumber{\pgfmathresult}%
				\let\pgfmatharga=\pgfmathresult
				\pgfmathfloatsqrt@{\pgfmatharga}%
				\pgfmathfloattosci@\pgfmathresult
				\pgfkeyslet{/pgfplots/table/@cell content}{\pgfmathresult}%
			\fi
		}%
	},
	/pgfplots/table/empty cells with/.style={%
		/pgfplots/table/postproc cell content/.append code={%
			\ifnum\pgfplotstablepartno=0
				\pgfkeysgetvalue{/pgfplots/table/@cell content}\pgfmathresult
				\ifx\pgfmathresult\pgfutil@empty
					\pgfkeyssetvalue{/pgfplots/table/@cell content}{#1}%
				\fi
			\fi
		}%
	},
	/pgfplots/table/fonts by sign/.style 2 args={%
		/pgfplots/table/postproc cell content/.append code={%
			\pgfkeysgetvalue{/pgfplots/table/@cell content}\pgfplotsretval
			\ifx\pgfplotsretval\pgfutil@empty
			\else
				\t@pgfplots@toka=\expandafter{\pgfplotsretval}%
				\t@pgfplots@tokb={#1}%
				\t@pgfplots@tokc={#2}%
				\pgfmathfloatparsenumber{\pgfkeysvalueof{/pgfplots/table/@preprocessed cell content}}%
				\pgfmathfloatifflags{\pgfmathresult}{-}{%
					\edef\pgfmathresult{{\the\t@pgfplots@tokc{\the\t@pgfplots@toka}}}%
				}{%
					\edef\pgfmathresult{{\the\t@pgfplots@tokb{\the\t@pgfplots@toka}}}%
				}%
				\pgfkeyslet{/pgfplots/table/@cell content}{\pgfmathresult}%
			\fi
		}%
	},%
	%
	/pgfplots/table/font/.initial=,
	/pgfplots/table/.search also={/pgf/number format,/pgfplots/table/create col},
	%--------------------------------------------------
	% /pgfplots/table/.unknown/.code={%
	% 	\let\pgfplots@table@curkeyname=\pgfkeyscurrentname
	% 	\pgfqkeys{/pgf/number format}{\pgfplots@table@curkeyname=#1}%
	% },%
	%-------------------------------------------------- 
	/pgfplots/table/create col/assign first/.style={
		/pgfplots/table/create col/assign%
	},
	/pgfplots/table/create col/assign last/.style={
		/pgfplots/table/create col/assign%
	},
	/pgfplots/table/create col/assign/.style={
		/pgfplots/table/create col/next content={}%
	},
	/pgfplots/table/create col/next content/.initial={},
	/pgfplots/table/create col/copy/.style={%
		/pgfplots/table/create col/assign/.code={%
			\getthisrow{#1}\pgfmathresult
			\pgfkeyslet{/pgfplots/table/create col/next content}\pgfmathresult%
		}%
	},
	/pgfplots/table/create col/set/.style={%
		/pgfplots/table/create col/assign/.code={%
			\def\pgfmathresult{#1}%
			\pgfkeyslet{/pgfplots/table/create col/next content}\pgfmathresult%
		}%
	},%
	/pgfplots/table/create col/set list/.code={%
		\pgfplots@assign@list\pgfmathaccumb{#1}%
		\pgfkeysalso{/pgfplots/table/create col/@from list struct=\pgfmathaccumb}%
	},%
	/pgfplots/table/create col/expr accum/.code 2 args={%
		\ifpgfplots@usefpu
			\pgfkeysalso{/pgf/fpu=true,/pgf/fpu/output format=sci}%
		\fi
		\pgfkeysdef{/pgfplots/table/create col/assign}{%
			\ifx\pgfmathaccuma\pgfutil@empty
				\pgfmathparse{#2}%
				\let\pgfmathaccuma=\pgfmathresult
			\fi
			\pgfmathparse{#1}%
			\pgfkeyslet{/pgfplots/table/create col/next content}\pgfmathresult%
			\let\pgfmathaccuma=\pgfmathresult
		}%
	},
	/pgfplots/table/create col/expr/.style={%
		/pgfplots/table/create col/expr accum={#1}{0}%
	},%
	/pgfplots/table/create col/copy column from table/.code 2 args={%
		\pgfplotstablegetcolumn{#2}\of{#1}\to\pgfmathaccumb
		\pgfkeysalso{/pgfplots/table/create col/@from list struct=\pgfmathaccumb}%
	},
	/pgfplots/table/create col/@from list struct/.code={%
		\pgfplotslistcopy#1\to\pgfmathaccumb
		\pgfkeysdef{/pgfplots/table/create col/assign}{%
			\pgfplotslistcheckempty\pgfmathaccumb
			\ifpgfplotslistempty
				\pgfkeyslet{/pgfplots/table/create col/next content}\pgfutil@empty%
			\else
				\pgfplotslistpopfront\pgfmathaccumb\to\pgfmathresult
				\pgfkeyslet{/pgfplots/table/create col/next content}\pgfmathresult%
			\fi
		}%
	},
	/pgfplots/table/create col/linear regression/.code={%
		\pgfplotstable@linear@regression{#1}%
		\pgfkeysalso{/pgfplots/table/create col/@from list struct=\pgfplotsretval}%
	},
	/pgfplots/table/create col/linear regression/.default=,%
	/pgfplots/table/create col/linear regression/x/.initial=,%
	/pgfplots/table/create col/linear regression/y/.initial=,%
	/pgfplots/table/create col/linear regression/table/.initial=,%
	/pgfplots/table/create col/linear regression/variance/.initial=,%
	/pgfplots/table/create col/linear regression/variance list/.initial=,%
	/pgfplots/table/create col/linear regression/variance src/.initial=,%
	/pgfplots/table/create col/linear regression/xmode/.initial=,% auto
	/pgfplots/table/create col/linear regression/ymode/.initial=,% auto
	/pgfplots/table/create col/quotient/.style={%
		/pgfplots/table/columns={#1},
		/pgfplots/table/create col/assign first/.style={%
			/pgfplots/table/create col/next content=
		},%
		/pgfplots/table/create col/assign/.code={%
			\pgfmathfloatparsenumber{\prevrow{#1}}%
			\let\pgfmatharga=\pgfmathresult
			\pgfmathfloatparsenumber{\thisrow{#1}}%
			\let\pgfmathargb=\pgfmathresult
			\let\pgfmathaccuma=\pgfmathargb
			\pgfplots@ifzero\pgfmathargb{%
				\pgfkeyslet{/pgfplots/table/create col/next content}\pgfutil@empty%
			}{%
				\pgfmathfloatdivide@{\pgfmatharga}{\pgfmathargb}%
				\expandafter\pgfmathfloattosci@\expandafter{\pgfmathresult}%
				\pgfkeyslet{/pgfplots/table/create col/next content}\pgfmathresult%
			}%
		}%
	},%
	/pgfplots/table/create col/iquotient/.style={%
		/pgfplots/table/columns={#1},
		/pgfplots/table/create col/assign first/.style={%
			/pgfplots/table/create col/next content=
		},%
		/pgfplots/table/create col/assign/.code={%
			\pgfmathfloatparsenumber{\prevrow{#1}}%
			\let\pgfmathargb=\pgfmathresult
			\pgfmathfloatparsenumber{\thisrow{#1}}%
			\let\pgfmatharga=\pgfmathresult
			\let\pgfmathaccuma=\pgfmathargb
			\pgfplots@ifzero\pgfmathargb{%
				\pgfkeyslet{/pgfplots/table/create col/next content}\pgfutil@empty%
			}{%
				\pgfmathfloatdivide@{\pgfmatharga}{\pgfmathargb}%
				\expandafter\pgfmathfloattosci@\expandafter{\pgfmathresult}%
				\pgfkeyslet{/pgfplots/table/create col/next content}\pgfmathresult%
			}%
		}%
	},%
	%
	% Produces 'log2( \prevrow{#1}/\thisrow{#1} )
	%
	% Assumeing that every row contains error(h) = O(h^alpha)
	% and h_this = h_prev/2, this result in 'alpha', the convergence
	% rate.
	/pgfplots/table/create col/dyadic refinement rate/.style={%
		/pgfplots/table/columns={#1},
		/pgfplots/table/create col/assign first/.style={%
			/pgfplots/table/create col/next content=
		},%
		/pgfplots/table/create col/assign/.code={%
			\pgfmathfloatparsenumber{\prevrow{#1}}%
			\let\pgfmatharga=\pgfmathresult
			\pgfmathfloatparsenumber{\thisrow{#1}}%
			\let\pgfmathargb=\pgfmathresult
			\let\pgfmathaccuma=\pgfmathargb
			\pgfplots@ifzero\pgfmathargb{%
				\pgfkeyslet{/pgfplots/table/create col/next content}\pgfutil@empty%
			}{%
				\pgfmathfloatdivide@{\pgfmatharga}{\pgfmathargb}%
				\pgfmathlog@float{\pgfmathresult}%
				\ifx\pgfmathresult\pgfutil@empty
				\else
					\pgfmathmultiply@{1.442695}{\pgfmathresult}%
				\fi
				\pgfkeyslet{/pgfplots/table/create col/next content}\pgfmathresult%
			}%
		}%
	},%
	/pgfplots/table/create col/idyadic refinement rate/.style={%
		/pgfplots/table/columns={#1},
		/pgfplots/table/create col/assign first/.style={%
			/pgfplots/table/create col/next content=
		},%
		/pgfplots/table/create col/assign/.code={%
			\pgfmathfloatparsenumber{\prevrow{#1}}%
			\let\pgfmathargb=\pgfmathresult
			\pgfmathfloatparsenumber{\thisrow{#1}}%
			\let\pgfmatharga=\pgfmathresult
			\let\pgfmathaccuma=\pgfmathargb
			\pgfplots@ifzero\pgfmathargb{%
				\pgfkeyslet{/pgfplots/table/create col/next content}\pgfutil@empty%
			}{%
				\pgfmathfloatdivide@{\pgfmatharga}{\pgfmathargb}%
				\pgfmathlog@float{\pgfmathresult}%
				\ifx\pgfmathresult\pgfutil@empty
				\else
					\pgfmathmultiply@{1.442695}{\pgfmathresult}%
				\fi
				\pgfkeyslet{/pgfplots/table/create col/next content}\pgfmathresult%
			}%
		}%
	},%
	/pgfplots/table/create col/gradient/.style 2 args={%
		/pgfplots/table/columns={#1,#2},
		/pgfplots/table/create col/assign first/.code={%
			\pgfmathfloatparsenumber{\thisrow{#1}}%
			\let\pgfmathaccuma=\pgfmathresult
			\pgfmathfloatparsenumber{\thisrow{#2}}%
			\let\pgfmathaccumb=\pgfmathresult
			\def\pgfmathresult{}% leave first empty.
			\pgfkeyslet{/pgfplots/table/create col/next content}\pgfmathresult%
		},%
		/pgfplots/table/create col/assign/.code={%
			\let\pgfmathcur@x=\pgfmathaccuma
			\let\pgfmathcur@y=\pgfmathaccumb
			\pgfmathfloatparsenumber{\thisrow{#1}}%
			\let\pgfmathnext@x=\pgfmathresult
			\let\pgfmathaccuma=\pgfmathnext@x
			\pgfmathfloatparsenumber{\thisrow{#2}}%
			\let\pgfmathnext@y=\pgfmathresult
			\let\pgfmathaccumb=\pgfmathnext@y
			\pgfmathfloatsubtract@{\pgfmathnext@x}{\pgfmathcur@x}%
			\let\pgfmathdiff@x=\pgfmathresult
			\pgfmathfloatsubtract@{\pgfmathnext@y}{\pgfmathcur@y}%
			\let\pgfmathdiff@y=\pgfmathresult
			\pgfplots@ifzero\pgfmathdiff@x{%
				\pgfkeyslet{/pgfplots/table/create col/next content}\pgfutil@empty%
			}{%
				\pgfmathfloatdivide@{\pgfmathdiff@y}{\pgfmathdiff@x}%
				\expandafter\pgfmathfloattosci@\expandafter{\pgfmathresult}%
				\pgfkeyslet{/pgfplots/table/create col/next content}\pgfmathresult%
			}%
		},%
	},%
	/pgfplots/table/create col/gradient loglog/.style 2 args={%
		/pgfplots/table/columns={#1,#2},
		/pgfplots/table/create col/assign first/.code={%
			\pgfmathlog{\thisrow{#1}}%
			\let\pgfmathaccuma=\pgfmathresult
			\pgfmathlog{\thisrow{#2}}%
			\let\pgfmathaccumb=\pgfmathresult
			\def\pgfmathresult{}% leave first empty.
			\pgfkeyslet{/pgfplots/table/create col/next content}\pgfmathresult%
		},%
		/pgfplots/table/create col/assign/.code={%
			\let\pgfmathcur@x=\pgfmathaccuma
			\let\pgfmathcur@y=\pgfmathaccumb
			\pgfmathlog{\thisrow{#1}}%
			\let\pgfmathnext@x=\pgfmathresult
			\let\pgfmathaccuma=\pgfmathresult
			\pgfmathlog{\thisrow{#2}}%
			\let\pgfmathnext@y=\pgfmathresult
			\let\pgfmathaccumb=\pgfmathresult
			\pgfplots@loop@CONTINUEtrue
			\ifx\pgfmathcur@x\pgfutil@empty		\pgfplots@loop@CONTINUEfalse\fi
			\ifx\pgfmathcur@y\pgfutil@empty		\pgfplots@loop@CONTINUEfalse\fi
			\ifx\pgfmathnext@x\pgfutil@empty	\pgfplots@loop@CONTINUEfalse\fi
			\ifx\pgfmathnext@y\pgfutil@empty	\pgfplots@loop@CONTINUEfalse\fi
			\ifpgfplots@loop@CONTINUE
				\pgfmathsubtract@{\pgfmathnext@x}{\pgfmathcur@x}%
				\let\pgfmathdiff@x=\pgfmathresult
				\pgfmathsubtract@{\pgfmathnext@y}{\pgfmathcur@y}%
				\let\pgfmathdiff@y=\pgfmathresult
				% FPU is more robust:
				\pgfmathfloatparsenumber\pgfmathdiff@x\let\pgfmathdiff@x=\pgfmathresult
				\pgfmathfloatparsenumber\pgfmathdiff@y\let\pgfmathdiff@y=\pgfmathresult
				\pgfplots@ifzero\pgfmathdiff@x{%
					\pgfkeyslet{/pgfplots/table/create col/next content}\pgfutil@empty%
				}{%
					\pgfmathfloatdivide@{\pgfmathdiff@y}{\pgfmathdiff@x}%
					\expandafter\pgfmathfloattosci@\expandafter{\pgfmathresult}%
					\pgfkeyslet{/pgfplots/table/create col/next content}\pgfmathresult%
				}%
			\else
			%	\pgfmathfloatcreate{3}{0.0}{0}%
			%	\pgfmathfloattosci@\pgfmathresult
			%	\pgfkeyslet{/pgfplots/table/create col/next content}\pgfmathresult%
				\pgfkeyslet{/pgfplots/table/create col/next content}\pgfutil@empty%
			\fi
		},%
	},%
	/pgfplots/table/create col/gradient semilogx/.style 2 args={%
		/pgfplots/table/columns={#1,#2},
		/pgfplots/table/create col/assign first/.code={%
			\pgfmathfloatparsenumber{\thisrow{#1}}
			\pgfmathfloatln@{\pgfmathresult}%
			\let\pgfmathaccuma=\pgfmathresult
			\pgfmathfloatparsenumber{\thisrow{#2}}%
			\let\pgfmathaccumb=\pgfmathresult
			\def\pgfmathresult{}% leave first empty.
			\pgfkeyslet{/pgfplots/table/create col/next content}\pgfmathresult%
		},%
		/pgfplots/table/create col/assign/.code={%
			\let\pgfmathcur@x=\pgfmathaccuma
			\let\pgfmathcur@y=\pgfmathaccumb
			\pgfmathfloatparsenumber{\thisrow{#1}}
			\pgfmathfloatln@{\pgfmathresult}%
			\let\pgfmathnext@x=\pgfmathresult
			\let\pgfmathaccuma=\pgfmathresult
			%
			\pgfmathfloatparsenumber{\thisrow{#2}}%
			\let\pgfmathnext@y=\pgfmathresult
			\let\pgfmathaccumb=\pgfmathresult
			\pgfmathfloatsubtract@{\pgfmathnext@x}{\pgfmathcur@x}%
			\let\pgfmathdiff@x=\pgfmathresult
			\pgfmathfloatsubtract@{\pgfmathnext@y}{\pgfmathcur@y}%
			\let\pgfmathdiff@y=\pgfmathresult
			\pgfplots@ifzero\pgfmathdiff@x{%
				\pgfkeyslet{/pgfplots/table/create col/next content}\pgfutil@empty%
			}{%
				\pgfmathfloatdivide@{\pgfmathdiff@y}{\pgfmathdiff@x}%
				\expandafter\pgfmathfloattosci@\expandafter{\pgfmathresult}%
				\pgfkeyslet{/pgfplots/table/create col/next content}\pgfmathresult%
			}%
		},%
	},%
	/pgfplots/table/create col/gradient semilogy/.style 2 args={%
		/pgfplots/table/columns={#1,#2},
		/pgfplots/table/create col/assign first/.code={%
			\pgfmathfloatparsenumber{\thisrow{#1}}%
			\let\pgfmathaccuma=\pgfmathresult
			\pgfmathfloatparsenumber{\thisrow{#2}}
			\pgfmathfloatln@{\pgfmathresult}%
			\let\pgfmathaccumb=\pgfmathresult
			\def\pgfmathresult{}% leave first empty.
			\pgfkeyslet{/pgfplots/table/create col/next content}\pgfmathresult%
		},%
		/pgfplots/table/create col/assign/.code={%
			\let\pgfmathcur@x=\pgfmathaccuma
			\let\pgfmathcur@y=\pgfmathaccumb
			\pgfmathfloatparsenumber{\thisrow{#2}}
			\pgfmathfloatln@{\pgfmathresult}%
			\let\pgfmathnext@y=\pgfmathresult
			\let\pgfmathaccumb=\pgfmathresult
			%
			\pgfmathfloatparsenumber{\thisrow{#1}}%
			\let\pgfmathnext@x=\pgfmathresult
			\let\pgfmathaccuma=\pgfmathresult
			\pgfmathfloatsubtract@{\pgfmathnext@x}{\pgfmathcur@x}%
			\let\pgfmathdiff@x=\pgfmathresult
			\pgfmathfloatsubtract@{\pgfmathnext@y}{\pgfmathcur@y}%
			\let\pgfmathdiff@y=\pgfmathresult
			\pgfplots@ifzero\pgfmathdiff@x{%
				\pgfkeyslet{/pgfplots/table/create col/next content}\pgfutil@empty%
			}{%
				\pgfmathfloatdivide@{\pgfmathdiff@y}{\pgfmathdiff@x}%
				\expandafter\pgfmathfloattosci@\expandafter{\pgfmathresult}%
				\pgfkeyslet{/pgfplots/table/create col/next content}\pgfmathresult%
			}%
		},%
	},%
	/pgfplots/table/typeset cell/.code={%
		\ifnum\c@pgfplotstable@colindex=\c@pgfplotstable@numcols\relax
			\pgfkeyssetvalue{/pgfplots/table/@cell content}{#1\\}%
		\else
			\pgfkeyssetvalue{/pgfplots/table/@cell content}{#1&}%
		\fi
	},
	% this can be omitted to omit the header row:
	/pgfplots/table/output empty row/.style={%
		/pgfplots/table/typeset cell/.code={%
			\pgfkeyslet{/pgfplots/table/@cell content}\pgfutil@empty%
		}%
	},
	/pgfplots/table/reset styles/.style={
		/pgfplots/table/every table/.code=,
		/pgfplots/table/every odd column/.code=,
		/pgfplots/table/every even column/.code=,
		/pgfplots/table/every first column/.code=,
		/pgfplots/table/every last column/.code=,
		/pgfplots/table/every head row/.code=,
		/pgfplots/table/every odd row/.code=,
		/pgfplots/table/every even row/.code=,
		/pgfplots/table/every first row/.code=,
		/pgfplots/table/every last row/.code=,
		/pgfplots/table/postproc cell content/.code=,
		/pgfplots/table/preproc cell content/.code=,
	},
	/pgfplots/table/colnames from/.initial=,% for \pgfplotstabletranspose
	/pgfplots/table/input colnames to/.initial=colnames,% for \pgfplotstabletranspose
	/pgfplots/table/sort/.is if=pgfplotstable@sort,
	/pgfplots/table/sort/.default=true,
	/pgfplots/table/sort key/.initial=[index]0,%
	% the argument of `sort cmp' will be evaluates as style in the key
	% path /pgfplots/. See pgfplotsutil.code.tex for available
	% styles.
	/pgfplots/table/sort cmp/.initial=float <,
	/pgfplots/table/sort key from/.initial=,
}
% 'function graph cut y'={<epsilon>}{<options>}{comma-separated-list of specs where to get yi}
%
% fills the column with x1,...,xN such that yi(xi) == epsilon where
%
% In other words, it computes cuts points between the line 
% y == epsilon and one or more other plots yi(x) and returns the 'x'
% values of the cuts.
%
% Example:
% \pgfplotstableset{
%	create on use/cut/.style={create col/function graph cut y={7e-4}{x=Basis,ymode=log,xmode=log}{{table=regtable,y=special-L2}}},
%}
\pgfkeysdefnargs{/pgfplots/table/create col/function graph cut y}{3}{\pgfplotstable@fgc@init{#1}{#2}{#3}{y}{x}}
\pgfkeysdefnargs{/pgfplots/table/create col/function graph cut x}{3}{\pgfplotstable@fgc@init{#1}{#2}{#3}{x}{y}}
\pgfkeys{%
	/pgfplots/table/create col/function graph cut/xmode/.is choice,
	/pgfplots/table/create col/function graph cut/xmode/linear/.code={\def\pgfplotstable@fgc@xmode{0}},%
	/pgfplots/table/create col/function graph cut/xmode/log/.code={\def\pgfplotstable@fgc@xmode{1}},%
	/pgfplots/table/create col/function graph cut/ymode/.is choice,
	/pgfplots/table/create col/function graph cut/ymode/linear/.code={\def\pgfplotstable@fgc@ymode{0}},%
	/pgfplots/table/create col/function graph cut/ymode/log/.code={\def\pgfplotstable@fgc@ymode{1}},%
	/pgfplots/table/create col/function graph cut/x/.initial=,
	/pgfplots/table/create col/function graph cut/y/.initial=,
	/pgfplots/table/create col/function graph cut/table/.initial=,
	% foreach={\d in {1,2,3,4}}{{table\d}}
	/pgfplots/table/create col/function graph cut/foreach/.initial=,
}


\pgfkeyslet{/pgfplots/table/TeX comment}\pgfplots@PERCENT@TEXT
\pgfkeysgetvalue{/pgfplots/table/postproc cell content/.@cmd}\pgfplotstable@postproccellcontent@EMPTY


% A helper macro to automatically remove the "hangover" created by
% 'colortbl'.
% This allows compatibility between my 'sci sep align' and 'dec sep align' 
% implementations and \rowcolor. Otherwise, the hangover
% would overwrite digits near the separator.
%
% @remark This does also work if colortbl is not loaded.
\def\pgfplotstableresetcolortbloverhangright{%
	\pgfutil@ifundefined{CT@row@color}{\relax}{%
		\global\let\pgfplots@origrowcolorcmd=\CT@row@color
		\gdef\CT@row@color{%
			\pgfplots@origrowcolorcmd
			\@tempdimc=0pt
			\global\let\CT@row@color=\pgfplots@origrowcolorcmd
		}%
	}%
}%
\def\pgfplotstableresetcolortbloverhangleft{%
	\pgfutil@ifundefined{CT@row@color}{\relax}{%
		\global\let\pgfplots@origrowcolorcmd=\CT@row@color
		\gdef\CT@row@color{%
			\pgfplots@origrowcolorcmd
			\@tempdimb=0pt
			\global\let\CT@row@color=\pgfplots@origrowcolorcmd
		}%
	}%
}%


% \pgfplotstablesave[<options>]{<\tablename>}{file name}
\def\pgfplotstablesave{%
	\pgfutil@ifnextchar[{%
		\pgfplotstablesave@impl
	}{%
		\pgfplotstablesave@impl[]%
	}%
}

\def\pgfplotstablesave@impl[#1]#2#3{%
	\pgfplotstabletypeset[%
		reset styles,%
		disable rowcol styles,%
		begin table={},%
		end table={},%
		typeset cell/.code={%
			\begingroup
			\t@pgfplots@toka={##1}%
			\ifcase\pgfplotstableread@OUTCOLSEP@CASE\relax
				% col sep=SPACE:
				\t@pgfplots@tokb=\expandafter{\pgfplotstableread@tab}%
				\pgfplots@ifempty{##1}{%
					\t@pgfplots@toka={{}}%
				}{}%
				\xdef\pgfplots@glob@TMPc{\the\t@pgfplots@toka\the\t@pgfplots@tokb}%
			\or
				% col sep=comma:
				\t@pgfplots@tokb={,}%
				\xdef\pgfplots@glob@TMPc{\the\t@pgfplots@toka\the\t@pgfplots@tokb}%
			\or
				% col sep=semicolon:
				\t@pgfplots@tokb={;}%
				\xdef\pgfplots@glob@TMPc{\the\t@pgfplots@toka\the\t@pgfplots@tokb}%
			\or
				% col sep=colon:
				\t@pgfplots@tokb={:}%
				\xdef\pgfplots@glob@TMPc{\the\t@pgfplots@toka\the\t@pgfplots@tokb}%
			\or
				% col sep=braces:
				\xdef\pgfplots@glob@TMPc{{\the\t@pgfplots@toka}}%
			\or
				% col sep=tab:
				\xdef\pgfplots@glob@TMPc{\the\t@pgfplots@toka\pgfplotstableread@tab}%
			\or
				% col sep=&:
				\xdef\pgfplots@glob@TMPc{\the\t@pgfplots@toka&}%
			\fi
			\endgroup
			\pgfkeyslet{/pgfplots/table/@cell content}\pgfplots@glob@TMPc%
		},%
		before row=,%
		after row=,%
		skip coltypes,%
		typeset=false,%
		string type,%
		TeX comment=,%
		columns=,%
		font=,%
		/pgfplots/table/col sep/.is choice,%
		/pgfplots/table/col sep/space/.code		= {\def\pgfplotstableread@OUTCOLSEP@CASE{0}},%
		/pgfplots/table/col sep/comma/.code		= {\def\pgfplotstableread@OUTCOLSEP@CASE{1}},%
		/pgfplots/table/col sep/semicolon/.code	= {\def\pgfplotstableread@OUTCOLSEP@CASE{2}},%
		/pgfplots/table/col sep/colon/.code		= {\def\pgfplotstableread@OUTCOLSEP@CASE{3}},%
		/pgfplots/table/col sep/braces/.code	= {\def\pgfplotstableread@OUTCOLSEP@CASE{4}},%
		/pgfplots/table/col sep/tab/.code		= {\def\pgfplotstableread@OUTCOLSEP@CASE{5}},%
		/pgfplots/table/col sep/&/.code			= {\def\pgfplotstableread@OUTCOLSEP@CASE{6}},%
		/pgfplots/table/col sep/ampersand/.code	= {\def\pgfplotstableread@OUTCOLSEP@CASE{6}},%
		/pgfplots/table/col sep=space,%
		/pgfplots/table/in col sep/.is choice,%
		/pgfplots/table/in col sep/space/.code		= {\def\pgfplotstableread@COLSEP@CASE{0}},%
		/pgfplots/table/in col sep/comma/.code		= {\def\pgfplotstableread@COLSEP@CASE{1}},%
		/pgfplots/table/in col sep/semicolon/.code	= {\def\pgfplotstableread@COLSEP@CASE{2}},%
		/pgfplots/table/in col sep/colon/.code		= {\def\pgfplotstableread@COLSEP@CASE{3}},%
		/pgfplots/table/in col sep/braces/.code		= {\def\pgfplotstableread@COLSEP@CASE{4}},%
		/pgfplots/table/in col sep/tab/.code		= {\def\pgfplotstableread@COLSEP@CASE{5}},%
		/pgfplots/table/in col sep/&/.code			= {\def\pgfplotstableread@COLSEP@CASE{6}},%
		/pgfplots/table/in col sep/ampersand/.code	= {\def\pgfplotstableread@COLSEP@CASE{6}},%
		/pgfplots/table/in col sep=space,%
		% WARNING: you NEED a '%' before '#1':
		#1,%
		/pgfplots/table/include outfiles=false,
		/pgfplots/table/outfile={#3}%
	]{#2}%
}%

% clears the table. 
\def\pgfplotstableclear#1{%
	\let#1=\relax
	\expandafter\let\csname \string#1@@table@name\endcsname=\relax
}%


% \pgfplotstablenew[<options>]{<numrows>}{<\name>}
% \pgfplotstablenew*[<options>]{<numrows>}{<\name>}
%
% Creates a new table from scratch. 
%
% The new table will contain all columns listed in the 'columns' key
% which must be present in <options>. The starred version
% \pgfplotstablenew* is not that strict: it will use the current value
% of the columns key (not matter where and when it has been set).
%
% Furthermore, there must be 'create on use' statements for every
% column which shall be generated. Columns are generated
% independently, in the order of appearance in 'columns'.
% The table will contain exactly <numrows> rows.
\def\pgfplotstablenew{%
	\begingroup
	\pgfutil@ifnextchar*{\pgfplotstablenew@star}{\pgfplotstablenew@nostar}}
\def\pgfplotstablenew@star*{\pgfutil@ifnextchar[{\pgfplotstablenew@impl}{\pgfplotstablenew@impl[]}}%
\def\pgfplotstablenew@nostar{%
	% reset columns key:
	\pgfkeyslet{/pgfplots/table/columns}{\pgfutil@empty}%
	\pgfutil@ifnextchar[{\pgfplotstablenew@impl}{\pgfplotstablenew@impl[]}}

\def\pgfplotstablenew@impl[#1]#2#3{%
	\ifx#3\pgfutil@undefined
	\else
		\ifx#3\relax
		\else
			% oh - there *is* already such a table. The 'getcolumnbyname' method suffers from a flaw in 'ifexists' that I do not want to fix right now.
			% To work around that flaw, I merely clear the old table here:
			% let's hope that '#3' really *was* a table and not some other junk...
			\pgfplotslistforeachungrouped#3\as\pgfplotstable@loc@TMPa{%
				\expandafter\let\csname\string#3@\pgfplotstable@loc@TMPa\endcsname=\relax%
			}%
		\fi
	\fi
	%
	\pgfplotsscanlinelengthinitzero
	% create a temporary column with the desired number of rows:
	\pgfutil@in@\pgfplotstablegetrowsof{#2}%
	\ifpgfutil@in@
		#2%
		\let\pgfplotstable@loc@TMPa=\pgfmathresult
	\else
		\def\pgfplotstable@loc@TMPa{#2}%
	\fi
	\pgfplotslistnew#3{@@@@@temporary@column@\\}%
	\expandafter\pgfplots@assign@list\csname\string#3@@@@@@temporary@column@\endcsname{1,2,...,\pgfplotstable@loc@TMPa}%
	%
	% now, create all real columns:
	\pgfplotstableset{#1,%
		/pgf/fpu/handlers/empty number/.code 2 args={%
			\pgfmathfloatcreate{0}{0.0}{0}%
		}%
	}%
	\pgfkeysgetvalue{/pgfplots/table/columns}{\pgfplotstable@colnames}%
	\ifx\pgfplotstable@colnames\pgfutil@empty
		\pgfplots@warning{\string\pgfplotstablenew[columns={},...]{#2}{\string#3} has been invoked - but empty tables are currently not really supported, sorry. You will have to live with an artifical column which contains temporary values.}%
	\else
		\expandafter\pgfplots@assign@list\expandafter\pgfplotstablenew@cols\expandafter{\pgfplotstable@colnames}%
		% make sure every requested column exists:
		\pgfutil@loop
		\pgfplotslistcheckempty\pgfplotstablenew@cols
		\ifpgfplotslistempty
			\pgfplots@loop@CONTINUEfalse
		\else
			\pgfplots@loop@CONTINUEtrue
		\fi
		\ifpgfplots@loop@CONTINUE
			\pgfplotslistpopfront\pgfplotstablenew@cols\to\pgfplotstablenew@col
			\expandafter\pgfplotstablegetcolumnbyname\expandafter{\pgfplotstablenew@col}\of#3\to\pgfplotstable@loc@TMPa
		\pgfutil@repeat
		% remove the temporary column:
		% FIXME this should be done after the '\fi'. But that will
		% lead to an error because empty tables are currently
		% unsupported!
		\pgfplotslistpopfront#3\to\pgfplotstable@loc@TMPa
	\fi
	\pgfplotsscanlinelengthcleanup
	\pgfplotstable@copy@to@globalbuffers#3{newlycreatedtable}%
	\endgroup
	\pgfplotstable@copy@globalbuffers@to#3%
}%

% \pgfplotstablevertcat{<table1>}{<table2>}
% appends the contents of <table2> to <table1>. To be more precise,
% only columns which exist already in <table1> will be used.
%
% If <table1> is undefined, <table2> will be copied completely to
% <table1>.
%
% #1 a table macro.
% #2 either a file name or a table macro.
\long\def\pgfplotstablevertcat#1#2{%
	\pgfplotstable@isloadedtable{#2}{%
		\pgfplotstable@isloadedtable{#1}{%
			% for each column in '#1':
			\pgfplotslistforeachungrouped#1\as\pgfplotstable@loc@TMPa{%
				% for each row in the corresponding column of '#2':
				\expandafter\pgfplotstablegetcolumnbyname\expandafter{\pgfplotstable@loc@TMPa}\of{#2}\to\pgfplotstable@loc@TMPb
				\pgfplotslistforeachungrouped\pgfplotstable@loc@TMPb\as\pgfplotstable@loc@TMPc{%
					\t@pgfplots@toka=\expandafter{\pgfplotstable@loc@TMPc}%
					\edef\pgfplotstable@loc@TMPd{%
						\noexpand\pgfplotslistpushback{\the\t@pgfplots@toka}\to\expandafter\noexpand\csname\string#1@\pgfplotstable@loc@TMPa\endcsname
					}%
					\pgfplotstable@loc@TMPd
				}%
			}%
		}{%
			\pgfplotstablecopy{#2}\to{#1}%
		}%
	}{%
		% FIXME : restore memory here !? SCOPING BUG
		\pgfplotstableread{#2}\pgfplotstable@tmptbl
		\pgfplotstablevertcat{#1}{\pgfplotstable@tmptbl}%
	}%
}

% \pgfplotstabletranspose{<\outtable>}{<intable>}
% \pgfplotstabletranspose[<colname>]{<\outtable>}{<intable>}
% Defines <\outtable> to be the transposed of <intable>. 
%
% If <colname> is not empty, the respective column's entries will be used to
% make output column names.
%
% #1: a table macro name which will be overwritten (redefined)
% #2: either a file name or a table macro (loaded table).
\def\pgfplotstabletranspose{%
	\begingroup
	\pgfutil@ifnextchar*{%
		\pgfplotstabletranspose@star%
	}{%
		\pgfkeyslet{/pgfplots/table/columns}\pgfutil@empty% clear!
		\pgfplotstabletranspose@star*%
	}%
}
\def\pgfplotstabletranspose@star*{\pgfutil@ifnextchar[{\pgfplotstabletranspose@opt}{\pgfplotstabletranspose@opt[]}}%
\long\def\pgfplotstabletranspose@opt[#1]#2#3{%
	\pgfplotstable@isloadedtable{#3}{%
		\pgfplotstabletranspose@[#1]{#2}{#3}%
	}{%
		\pgfplotstableread{#3}\pgfplotstable@tmptbl
		\pgfplotstabletranspose@[#1]{#2}{\pgfplotstable@tmptbl}%
	}%
}%


% Creates a new column named #1 and appends it to table #2.
%
% The column entries will be created using the command keys
% 'create col/assign' 
% 'create col/assign last'
%
% The key 'create col/assign' will be invoked for every row of table #2.
% It is supposed to assign the key 'create col/next content'.
% During evaluation of 'create col/assign', the macro '\thisrow{<col name>}' 
% expands to the current row's value of the column named by <col name>.
% Furthermore, '\nextrow{<col name>}' expands to the \emph{next} row's
% value of the designated column.
%
% Since the "next row" is not available if we are currently processing
% the last row, 'create col/assign last' is used in for the last row's
% value.
%
% You can use
% - \thisrow{<col name>}, 
% - \getthisrow{<col name>}{\macro}
% - \nextrow{<col name>}, 
% - \getnextrow{<col name>}{\macro}
%
% FIXME this documentation is incomplete. Please refer to pgfplotstable.pdf .
\def\pgfplotstablecreatecol{%
	\pgfutil@ifnextchar[{%
		\pgfplotstablecreatecol@opt
	}{%
		\pgfplotstablecreatecol@opt[]%
	}%
}%


% Typesets a table.
%
% \pgfplotstabletypeset[<options>]<\tablestructure>
% or
% \pgfplotstabletypeset[<options>]{<file name>}
%
% If you do not select any columns, the complete table is drawn.
%
% There are several options and styles which are available in
% <options>, see the declaration above.
%
% ATTENTION: the default implementation employs
% \begin{tabular}...\end{tabular} and is therefor only usable with
% LaTeX!
%
% You will need to reconfigure the tables.
%
% Inside of \pgfplotstabletypeset, the macros
% \pgfplotstablecol,\pgfplotstablecolname and
% \pgfplotstablerow will expand to the current column index, column
% name and row index, respectively.
\def\pgfplotstabletypeset{%
	\pgfutil@ifnextchar[{%	
		\pgfplotstabletypeset@opt
	}{%
		\pgfplotstabletypeset@opt[]%
	}%
}
\long\def\pgfplotstabletypeset@opt[#1]{%
	\begingroup
	\ifpgfplots@table@options@areset
	\else
		\pgfplotstableset{#1}%
	\fi
	\pgfplotstablecollectoneargwithpreparecatcodes{%
		\pgfplotstabletypeset@opt@collectarg[#1]%
	}%
}
\long\def\pgfplotstabletypeset@opt@collectarg[#1]#2{%
	\pgfplotstable@isloadedtable{#2}%
		{\pgfplotstabletypeset@opt@[#1]{#2}}%
		{\pgfplotstabletypesetfile@opt@[#1]{#2}}%
}


% Like \pgfplotstabletypeset, but the first argument is a file name.
% This is the same now; it will be recognised automatically.
\let\pgfplotstabletypesetfile=\pgfplotstabletypeset
\long\def\pgfplotstabletypesetfile@opt@[#1]#2{%
	%\begingroup <--- is already opened!
	\ifpgfplots@table@options@areset
	\else
		\pgfplotstableset@every@table{#2}{#1}%
		\pgfplots@table@options@aresettrue
	\fi
	\ifpgfplotstabletypeset@force@remake
		\pgfplotstabletypeset@includeoutfilesfalse
	\else
		\ifpgfplotstabletypeset@includeoutfiles
			\pgfplotstabletypeset@includeoutfilesfalse
			\pgfkeysgetvalue{/pgfplots/table/outfile}\pgfplotstable@outfilename
			\ifx\pgfplotstable@outfilename\pgfutil@empty
			\else
				\openin\r@pgfplots@reada=\pgfplotstable@outfilename\relax
				\ifeof\r@pgfplots@reada
				\else
					\pgfplotstabletypeset@includeoutfilestrue
				\fi
				\closein\r@pgfplots@reada
			\fi
		\fi
	\fi
	\ifpgfplotstabletypeset@includeoutfiles
		\input \pgfplotstable@outfilename\relax	
	\else
		\pgfplotstableread{#2}\pgfplotstabletypesetfile@opt@@
		\ifx\pgfplotstabletypesetfile@opt@@\relax
			% ERROR.
		\else
			\pgfplotstabletypeset\pgfplotstabletypesetfile@opt@@
		\fi
	\fi
	\pgfkeysgetvalue{/pgfplots/table/write to macro}\pgfplots@loc@TMPa
	\ifx\pgfplots@loc@TMPa\pgfutil@empty
		\def\pgfplots@loc@TMPa{\pgfutil@empty}%
	\else
		\expandafter\ifx\pgfplots@loc@TMPa\relax
			% is it really defined? NO! Sanity checking here:
			\def\pgfplots@loc@TMPa{\pgfutil@empty}%
		\fi
	\fi
	\expandafter\pgfmath@smuggleone\pgfplots@loc@TMPa
	\endgroup
}%

%%%%%%%%%%%%%%%%%%%%%%%%%%%%%%%%%%%%%%%%%%%%%%%%%%%%%%%%%%%%%%%%%%%%%%%%%%%
%
% IMPLEMENTATION
%
%%%%%%%%%%%%%%%%%%%%%%%%%%%%%%%%%%%%%%%%%%%%%%%%%%%%%%%%%%%%%%%%%%%%%%%%%%%


%%%%%%%%%%%%%%%


% #1: the original value
% #2: the output macro
\long\def\pgfplotstabletypeset@getfinalentry#1#2{%
	\begingroup
	\def\pgfkeysdefaultpath{/pgfplots/table/}%
	\def\pgfplotstablepartno{0}%
	\pgfkeyssetvalue{/pgfplots/table/@cell content}{#1}%
	\pgfkeyssetvalue{/pgfplots/table/@unprocessed cell content}{#1}%
	\ifpgfplotstable@disable@rowcolstyles
	\else
		\def\pgfplots@loc@TMPb##1{%
			\edef\pgfplotstable@loc@TMPa{/pgfplots/table/every row ##1\pgfplotstablerow\space column ##1\pgfplotstablecol}%
			\pgfkeysifdefined{\pgfplotstable@loc@TMPa/.@cmd}{%
				\expandafter\pgfplotstableset\expandafter{\pgfplotstable@loc@TMPa}%
			}{}%
			\edef\pgfplotstable@loc@TMPa{/pgfplots/table/every row ##1\pgfplotstablerow\space column \pgfplotstable@colname@for@styles}%
			\pgfkeysifdefined{\pgfplotstable@loc@TMPa/.@cmd}{%
				\expandafter\pgfplotstableset\expandafter{\pgfplotstable@loc@TMPa}%
			}{}%
		}%
		\pgfplots@loc@TMPb{}%
		% also accept the same with the 'row no' style:
		\pgfplots@loc@TMPb{no }%
		\pgfplotstable@debug@notify@cellcontent@afterrowcolstyles%
	\fi
	\pgfplotstable@debug@notify@cellcontent%
	\pgfkeysgetvalue{/pgfplots/table/@cell content}\pgfmathresult%
	\pgfkeyslet{/pgfplots/table/@cell content after rowcol styles}\pgfmathresult%
	\pgfkeysgetvalue{/pgfplots/table/preproc cell content/.@cmd}\pgfplotstable@assigncell
	\expandafter\pgfplotstable@assigncell\pgfmathresult\pgfeov
	\pgfkeysgetvalue{/pgfplots/table/@cell content}\pgfmathresult%
	\pgfkeyslet{/pgfplots/table/@preprocessed cell content}\pgfmathresult%
	\pgfplotstable@debug@notify@cellcontent@preprocessed%
	%
	\pgfkeysgetvalue{/pgfplots/table/assign cell content/.@cmd}\pgfplotstable@assigncell
	\expandafter\pgfplotstable@assigncell\pgfmathresult\pgfeov
	\pgfplotstable@debug@notify@cellcontent@assigned%
	%
	\pgfkeysgetvalue{/pgfplots/table/postproc cell content/.@cmd}\pgfplotstable@postproccellcontent
	\ifx\pgfplotstable@postproccellcontent\pgfplotstable@postproccellcontent@EMPTY
	\else
		% apply postprocessing to final cell content.
		\def\pgfplotstabletypeset@rawinput{#1}%
		%
		% This is complicated if there is an '&' in '@cell content',
		% so handle that specially!
		%
		% FIXME also support more than one '&' ?
		\pgfkeysgetvalue{/pgfplots/table/@cell content}\pgfmathresult%
		\expandafter\pgfutil@in@\expandafter&\expandafter{\pgfmathresult}%
		\ifpgfutil@in@
			\expandafter\pgfplotstabletypeset@postproc@separately\pgfmathresult\pgfplotstable@EOI
		\else
			\expandafter\pgfplotstable@postproccellcontent\pgfplotstabletypeset@rawinput\pgfeov
		\fi
	\fi
	\pgfplotstable@debug@notify@cellcontent@postprocessed%
	\pgfkeysgetvalue{/pgfplots/table/@cell content}\pgfmathresult%
	\pgfmath@smuggleone\pgfmathresult
	\endgroup
	\let#2=\pgfmathresult
}%

\def\pgfplotstable@debug@notify@cellcontent{}%
\def\pgfplotstable@debug@notify@cellcontent@preprocessed{}%
\def\pgfplotstable@debug@notify@cellcontent@afterrowcolstyles{}%
\def\pgfplotstable@debug@notify@cellcontent@assigned{}%
\def\pgfplotstable@debug@notify@cellcontent@postprocessed{}%
\def\pgfplotstable@debug@notify@preprocess@incol#1{}%
\def\pgfplotstable@debug@notify@preprocess@col#1{}%
\def\pgfplotstable@debug@notify@preprocessed@col#1{}%
\def\pgfplotstable@debug@notify@balancingcell{}%

\def\pgfplotstable@debug@notify@cellcontent@ACTIVE{%
	\begingroup
	\pgfkeysgetvalue{/pgfplots/table/@cell content}\pgfmathresult%
	\toks0=\expandafter{\pgfmathresult}%
	\immediate\write16{Row \pgfplotstablerow/out \the\c@pgfplotstable@counta: Before cell content processing: `\the\toks0'}%
	\endgroup
}%
\def\pgfplotstable@debug@notify@cellcontent@preprocessed@ACTIVE{%
	\begingroup
	\pgfkeysgetvalue{/pgfplots/table/@cell content}\pgfmathresult%
	\toks0=\expandafter{\pgfmathresult}%
	\immediate\write16{Row \pgfplotstablerow/out \the\c@pgfplotstable@counta: After '@preproc cell content  : `\the\toks0'}%
	\endgroup
}%
\def\pgfplotstable@debug@notify@cellcontent@afterrowcolstyles@ACTIVE{%
	\begingroup
	\pgfkeysgetvalue{/pgfplots/table/@cell content}\pgfmathresult%
	\toks0=\expandafter{\pgfmathresult}%
	\immediate\write16{Row \pgfplotstablerow/out \the\c@pgfplotstable@counta: After applying 'every row * column [no] *' styles : `\the\toks0'}%
	\endgroup
}%
\def\pgfplotstable@debug@notify@cellcontent@assigned@ACTIVE{%
	\begingroup
	\pgfkeysgetvalue{/pgfplots/table/@cell content}\pgfmathresult%
	\toks0=\expandafter{\pgfmathresult}%
	\immediate\write16{Row \pgfplotstablerow/out \the\c@pgfplotstable@counta: After 'assign cell content    : `\the\toks0'}%
	\endgroup
}%
\def\pgfplotstable@debug@notify@cellcontent@postprocessed@ACTIVE{%
	\begingroup
	\pgfkeysgetvalue{/pgfplots/table/@cell content}\pgfmathresult%
	\toks0=\expandafter{\pgfmathresult}%
	\immediate\write16{Row \pgfplotstablerow/out \the\c@pgfplotstable@counta: After 'postproc cell content  : `\the\toks0'}%
	\endgroup
}
\def\pgfplotstable@debug@notify@balancingcell@ACTIVE{%
	\immediate\write16{Row --/out \the\c@pgfplotstable@counta: Inserting empty cell to balance rows}%
}%
\def\pgfplotstable@debug@notify@preprocess@incol@ACTIVE#1{%
	\begingroup
	\toks0=\expandafter{#1}%
	\immediate\write16{>=== Loading input column `\the\toks0' <===}%
	\endgroup
}%
\def\pgfplotstable@debug@notify@preprocess@col@ACTIVE#1{%
	\begingroup
	\toks0=\expandafter{#1}%
	\immediate\write16{>=== Preprocessing output column `\the\toks0' <===}%
	\endgroup
}%
\def\pgfplotstable@debug@notify@preprocessed@col@ACTIVE#1{%
	\begingroup
	\toks0=\expandafter{#1}%
	\immediate\write16{>=== output column `\the\toks0' prepared, ok. <===}%
	\endgroup
}%
\def\pgfplotstable@debug@activate{%
	\ifnum\pgfkeysvalueof{/pgfplots/table/debug level}>0
		\let\pgfplotstable@debug@notify@preprocess@incol=\pgfplotstable@debug@notify@preprocess@incol@ACTIVE
		\let\pgfplotstable@debug@notify@preprocess@col=\pgfplotstable@debug@notify@preprocess@col@ACTIVE
		\let\pgfplotstable@debug@notify@preprocessed@col=\pgfplotstable@debug@notify@preprocessed@col@ACTIVE
		\let\pgfplotstable@debug@notify@cellcontent=\pgfplotstable@debug@notify@cellcontent@ACTIVE
		\let\pgfplotstable@debug@notify@cellcontent@preprocessed=\pgfplotstable@debug@notify@cellcontent@preprocessed@ACTIVE
		\let\pgfplotstable@debug@notify@cellcontent@afterrowcolstyles=\pgfplotstable@debug@notify@cellcontent@afterrowcolstyles@ACTIVE
		\let\pgfplotstable@debug@notify@cellcontent@assigned=\pgfplotstable@debug@notify@cellcontent@assigned@ACTIVE
		\let\pgfplotstable@debug@notify@cellcontent@postprocessed=\pgfplotstable@debug@notify@cellcontent@postprocessed@ACTIVE
		\let\pgfplotstable@debug@notify@balancingcell=\pgfplotstable@debug@notify@balancingcell@ACTIVE
	\fi
}%

% This routine invokes 'postproc cell content' for columns which
% contain the column separator '&'.
%
% #1&#2 is the formatted number, the result of 'dec sep align
% #3 is the (unformatted) input number.
\def\pgfplotstabletypeset@postproc@separately#1&#2\pgfplotstable@EOI{%
	\def\pgfmathresult{#1}%
	\pgfkeyslet{/pgfplots/table/@cell content}\pgfmathresult
	\expandafter\pgfplotstable@postproccellcontent\pgfplotstabletypeset@rawinput\pgfeov
	\pgfkeysgetvalue{/pgfplots/table/@cell content}\pgfplotstable@entry@a
	%
	\def\pgfplotstablepartno{1}%
	\def\pgfmathresult{#2}%
	\pgfkeyslet{/pgfplots/table/@cell content}\pgfmathresult
	\expandafter\pgfplotstable@postproccellcontent\pgfplotstabletypeset@rawinput\pgfeov
	\pgfkeysgetvalue{/pgfplots/table/@cell content}\pgfmathresult
	\t@pgfplots@toka=\expandafter{\pgfplotstable@entry@a}%
	\t@pgfplots@tokb=\expandafter{\pgfmathresult}%
	\edef\pgfmathresult{\the\t@pgfplots@toka&\the\t@pgfplots@tokb}%
	\pgfkeyslet{/pgfplots/table/@cell content}\pgfmathresult
}

% processes the option 'assign column name'
% FIXME : seems to be deprecated
\def\pgfplotstabletypeset@assign@final@colname#1#2{%
	\pgfkeysifdefined{/pgfplots/table/assign column name/.@cmd}{%
		\pgfkeyssetvalue{/pgfplots/table/column name}{#1}%
		\pgfkeysvalueof{/pgfplots/table/assign column name/.@cmd}#1\pgfeov
		\pgfkeysgetvalue{/pgfplots/table/column name}{#2}%
	}{}%
}
\def\pgfplotstabletypeset@nocolname{\pgfkeysnovalue}

% checks if #1 contains invalid chars for pgfkeys and sets
% \ifpgfutil@in@ to true if that is the case.
\def\pgfplotstable@checkspecialchars@pgfkeys#1\pgfplotstable@EOI{%
	\pgfutil@in@/{#1}%
	\ifpgfutil@in@
	\else
		\pgfutil@in@={#1}%
		\ifpgfutil@in@
		\else
			\pgfutil@in@,{#1}%
		\fi
	\fi
}%

% Upon execution, \pgfplotsretval contains a set of /pgfplots/table
% keys from 'every nth row' styles.
%
% @PRECONDITION this macro assumes it is run in \pgfplotstabletypeset.
\def\pgfplots@each@nth@styles{\let\pgfplotsretval\pgfutil@empty}%

% Defines an 'every nth row' style for an integer #1.
%
% #1: integer
% #2: style arguments
%
% All these styles are accumulated into the macro
% \pgfplots@each@nth@styles. 
\def\pgfplotstabletypeset@append@every@nth@row#1#2{%
	\edef\pgfplots@loc@TMPa{#1}%
	% chech for the special 'each nth row={3[+1]}{...} format:
	\expandafter\pgfutil@in@\expandafter[\expandafter{\pgfplots@loc@TMPa}%
	\ifpgfutil@in@
		\expandafter\pgfplotstabletypeset@append@every@nth@row@getshift\pgfplots@loc@TMPa%
		\t@pgfplots@toka={#2}%
		\edef\pgfplots@loc@TMPa{%
			\noexpand\pgfplotstabletypeset@append@every@nth@row@{\pgfplots@loc@TMPa}{\the\t@pgfplots@toka}{\pgfplots@loc@TMPb}%
		}%
		\pgfplots@loc@TMPa
	\else
		\expandafter\pgfplotstabletypeset@append@every@nth@row@\expandafter{\pgfplots@loc@TMPa}{#2}{0}%
	\fi
}%
\def\pgfplotstabletypeset@append@every@nth@row@getshift#1[#2]{%
	\def\pgfplots@loc@TMPa{#1}%
	\ifnum#2<0
		\c@pgf@countc=#1\relax
		\advance\c@pgf@countc by#2
		\edef\pgfplots@loc@TMPb{\the\c@pgf@countc}%
	\else
		\def\pgfplots@loc@TMPb{#2}%
	\fi
}%
% #3 is an additional shift. It is usually 0
\def\pgfplotstabletypeset@append@every@nth@row@#1#2#3{%
	\expandafter\def\expandafter\pgfplots@each@nth@styles\expandafter{%
		\pgfplots@each@nth@styles
		\expandafter\pgfplotsmathmodint\expandafter{\the\c@pgfplotstable@rowindex}{#1}%
		\ifnum\pgfmathresult=#3\relax
			\ifx\pgfplotsretval\pgfutil@empty
				\def\pgfplotsretval{#2}%
			\else
				\expandafter\def\expandafter\pgfplotsretval\expandafter{\pgfplotsretval,%
					#2}%
			\fi
		\fi
	}%
}

\def\pgfplotstable@insertemptycells@forbalance{%
	\pgfutil@loop
	\ifnum\c@pgfplotstable@counta<\pgfplotstable@firstnumrows\relax
		\pgfplotstable@debug@notify@balancingcell
		% let's hope @getfinalentry handles empty strings!
		\pgfplotstabletypeset@getfinalentry{}{\pgfplotstable@entry}%
		\expandafter\pgfplotslistpushback\pgfplotstable@entry\to\pgfplotstable@col@processed
		\pgfplotslistpushback\to\pgfplotstable@col@processed
		\advance\c@pgfplotstable@counta by1\relax
	\pgfutil@repeat
}

% TODO
% - replace grouped list foreach by popfront-loop and use arrays
%   directly -> group only the pgfkeys eval
\long\def\pgfplotstabletypeset@opt@[#1]#2{%
	%\begingroup <--- is already opened!
	%--------------------------------------------------
	% \pgfutil@ifundefined{#2}{%
	% 	\pgfplots@error{There is no such table '\string#2' loaded into memory. Maybe you meant to use '\string\pgfplotstabletypesetfile{\string#2}' instead of '\string\pgfplotstabletypeset{\string#2}'?}%
	% 	\pgfplotslistnewempty#2
	% }{}%
	%-------------------------------------------------- 
	\def\pgfplotstablename{#2}% the name of the actual table struct
	\def\pgfplotstablecolname{\pgfplotstable@colname}%
	\def\pgfplotstablecol{\the\c@pgfplotstable@colindex}%
	\def\pgfplotstablerow{\the\c@pgfplotstable@rowindex}%
	\def\pgfplotstablecols{\the\c@pgfplotstable@numcols}%
	\def\pgfplotstablerows{\the\c@pgfplotstable@numrows}%
	\ifpgfplots@table@options@areset
	\else
		\pgfplotstablegetname{#2}\pgfplotstable@loc@TMPa
		\expandafter\pgfplotstableset@every@table\expandafter{\pgfplotstable@loc@TMPa}{#1}%
	\fi
	\pgfkeysgetvalue{/pgfplots/table/outfile}\pgfplotstable@outfilename
	\pgfkeysgetvalue{/pgfplots/table/TeX comment}\pgfplots@TeX@comment
	\ifpgfplotstabletypeset@force@remake
		\pgfplotstabletypeset@includeoutfilesfalse
	\else
		\ifpgfplotstabletypeset@includeoutfiles
			\pgfplotstabletypeset@includeoutfilesfalse
			\ifx\pgfplotstable@outfilename\pgfutil@empty
			\else
				\pgfplots@logfileopen{\pgfplotstable@outfilename}%
				\openin\r@pgfplots@reada=\pgfplotstable@outfilename\relax
				\ifeof\r@pgfplots@reada \else\pgfplotstabletypeset@includeoutfilestrue \fi
				\closein\r@pgfplots@reada
			\fi
		\fi
	\fi
	\ifpgfplotstabletypeset@includeoutfiles
		\def\pgfplots@loc@TMPa{%
			\input \pgfplotstable@outfilename\relax
			\endgroup
		}%
	\else
		\def\pgfplots@loc@TMPa{%
			\pgfplotstabletypeset@opt@prepare{#2}%
		}%
	\fi
	\pgfplots@loc@TMPa
}%
\def\pgfplotstabletypeset@opt@prepare#1{%
	%
	% Prepare outfile and debug options:
	\let\pgfplotstable@notify@finished@line=\pgfutil@empty
	%
	% FLUSH assumes that \pgfplotstable@curline is finished. It
	% appends all its contents as-is to \pgfplotstable@result.
	%
	% Furthermore, it calls \pgfplotstable@notify@finished@line which
	% in turn may invoke additional output routines for the debug and
	% outfile options.
	%
	% Finally, it resets \pgfplotstable@curline.
	\def\pgfplotstable@curline@FLUSH{%
		\pgfplotstable@notify@finished@line
		\t@pgfplots@toka=\expandafter{\pgfplotstable@result}%
		\t@pgfplots@tokb=\expandafter{\pgfplotstable@curline}%
		\edef\pgfplotstable@result{\the\t@pgfplots@toka\the\t@pgfplots@tokb}%
		\let\pgfplotstable@curline=\pgfutil@empty
	}%
	\ifpgfplotstabletypesetdebug
		\immediate\write16{------- PGFPLOTSTABLE DEBUG MODE: --------}%
		\expandafter\def\expandafter\pgfplotstable@notify@finished@line\expandafter{\pgfplotstable@notify@finished@line
			\t@pgfplots@toka=\expandafter{\pgfplotstable@curline}%
			\immediate\write16{\the\t@pgfplots@toka\pgfplots@TeX@comment}%
		}%
		\pgfplotstable@debug@activate
	\fi
	\ifx\pgfplotstable@outfilename\pgfutil@empty
	\else
		\immediate\openout\pgfplotstable@outfile=\pgfplotstable@outfilename\relax
		\expandafter\def\expandafter\pgfplotstable@notify@finished@line\expandafter{\pgfplotstable@notify@finished@line
			\t@pgfplots@toka=\expandafter{\pgfplotstable@curline}%
			\immediate\write\pgfplotstable@outfile{\the\t@pgfplots@toka\pgfplots@TeX@comment}%
		}%
	\fi
	%
	% Start operation:
	\pgfkeysgetvalue{/pgfplots/table/columns}{\pgfplotstable@colnames}%
	\ifx\pgfplotstable@colnames\pgfutil@empty
		\pgfplotstablegetcolumnlist#1\to\pgfplotstable@colnames
	\else
		\expandafter\pgfplotslistnew\expandafter\pgfplotstable@colnames\expandafter{\pgfplotstable@colnames}%
	\fi
	%
	%
	\ifpgfplotstable@sort
		% make sure any columns exist (especially create on use). 
		% this can be done by calling getcolumnbyname once for every
		% column:
		\pgfplotslistforeachungrouped\pgfplotstable@colnames\as\pgfplotstable@colname{%
			\pgfplotstable@is@colname\pgfplotstable@colname
			\ifpgfplotstableread@foundcolnames
			\else
				\pgfplotstablegetcolumnnamebyindex\pgfplotstable@colname\of#1\to\pgfplotstable@colname
			\fi
			\expandafter\pgfplotstablegetcolumnbyname\expandafter{\pgfplotstable@colname}\of#1\to\pgfplotstable@col
		}%
		\let\pgfplotstable@colname=\relax
		\let\pgfplotstable@col=\relax
		%
		\def\pgfplots@loc@TMPa{%
			\pgfplotstablesort\pgfplotstable@sortedtbl{#1}%
			\pgfplotstabletypeset@opt@@{\pgfplotstable@sortedtbl}%
		}%
	\else
		\def\pgfplots@loc@TMPa{%
			\pgfplotstabletypeset@opt@@{#1}%
		}%
	\fi
	\pgfplots@loc@TMPa
}
\def\pgfplotstabletypeset@opt@@#1{%
	\global\pgfplotslistnewempty\pgfplotstabletypeset@final@colnames
	\global\pgfplotslistnewempty\pgfplotstabletypeset@final@coltypes
	\global\pgfplotslistnewempty\pgfplotstabletypeset@final@cols
	\let\c@pgfplotstable@numcols=\c@pgf@counta
	\let\c@pgfplotstable@numrows=\c@pgf@countd
	\let\c@pgfplotstable@rowindex=\c@pgf@countc
	\let\c@pgfplotstable@colindex=\c@pgf@countb
	\pgfplotslistsize\pgfplotstable@colnames\to\c@pgfplotstable@numcols
	\def\pgfplotstable@firstnumrows{-1}%
	\c@pgfplotstable@numrows=-1\relax
	\c@pgfplotstable@colindex=0\relax
	% FOREACH COLUMN:
	\pgfplotslistforeach\pgfplotstable@colnames\as\pgfplotstable@colname{%
		\pgfplotstable@debug@notify@preprocess@incol\pgfplotstable@colname
		%
		\c@pgfplotstable@rowindex=0\relax
		\pgfplotstable@is@colname\pgfplotstable@colname
		\ifpgfplotstableread@foundcolnames
		\else
			\pgfplotstablegetcolumnnamebyindex\pgfplotstable@colname\of#1\to\pgfplotstable@colname
		\fi
		\expandafter\pgfplotstablegetcolumnbyname\expandafter{\pgfplotstable@colname}\of#1\to\pgfplotstable@col
		%
		% Init number of *input* rows here. This may not be the same
		% as the number of *output* rows (see the row predicate
		% below).
		%
		% Accessable with \pgfplotstablerows in style keys.
		\ifnum\c@pgfplotstable@numrows=-1\relax
			\pgfplotslistsize\pgfplotstable@col\to\c@pgfplotstable@numrows
			\global\c@pgfplotstable@numrows=\c@pgfplotstable@numrows
		\fi
		%
		%
		% %%%%%%%%%%%%%%%%%%%%%%%%%%%%%%%%%%%%%%%%%%%%%%%%%%%%%%%%%%%%%%%%%%%%%%%%%%%%%%%%
		% Set keys for columns!
		\ifodd\c@pgfplotstable@colindex
			\t@pgfplots@toka={every odd column}%
		\else
			\t@pgfplots@toka={every even column}%
		\fi
		\ifnum\c@pgfplotstable@colindex=0\relax
			\t@pgfplots@toka=\expandafter{\the\t@pgfplots@toka,every first column}%
		\fi
		% save this now before we increment '\c@pgfplotstable@colindex':
		\edef\pgfplotstable@displaycolkey{display columns/\the\c@pgfplotstable@colindex/.try,every col no \the\c@pgfplotstable@colindex/.try}%
		%
		\global\advance\c@pgfplotstable@colindex by1\relax
		\ifnum\c@pgfplotstable@colindex=\c@pgfplotstable@numcols
			\t@pgfplots@toka=\expandafter{\the\t@pgfplots@toka,every last column}%
		\fi
		% temporarily restore it: we may need it in row predicates:
		\global\advance\c@pgfplotstable@colindex by-1\relax
		\ifpgfplotstable@disable@rowcolstyles
			% ok, then don't check for 'columns/<name>' and 
			% 'display columns/<index>':
			\edef\pgfplotstable@loc@TMPa{\the\t@pgfplots@toka}%
		\else
			\t@pgfplots@tokb=\expandafter{\pgfplotstable@colname}%
			\expandafter\pgfplotstable@checkspecialchars@pgfkeys\the\t@pgfplots@tokb\pgfplotstable@EOI
			\ifpgfutil@in@
				\edef\pgfplotstable@colname@for@styles{{\the\t@pgfplots@tokb}}%
			\else
				\edef\pgfplotstable@colname@for@styles{\the\t@pgfplots@tokb}%
			\fi
			\edef\pgfplotstable@loc@TMPa{\the\t@pgfplots@toka,columns/\pgfplotstable@colname@for@styles/.try}%
			\t@pgfplots@toka=\expandafter{\pgfplotstable@loc@TMPa}%
			\t@pgfplots@tokb=\expandafter{\pgfplotstable@displaycolkey}%
			\edef\pgfplotstable@loc@TMPa{\the\t@pgfplots@toka,\the\t@pgfplots@tokb}%
		\fi
		\expandafter\pgfplotstableset\expandafter{\pgfplotstable@loc@TMPa}%
		% %%%%%%%%%%%%%%%%%%%%%%%%%%%%%%%%%%%%%%%%%%%%%%%%%%%%%%%%%%%%%%%%%%%%%%%%%%%%%%%%
		%
		\pgfkeysgetvalue{/pgfplots/table/column name}{\pgfplotstable@colname@out}%
		\ifx\pgfplotstable@colname@out\pgfplotstabletypeset@nocolname
			\let\pgfplotstable@colname@out=\pgfplotstable@colname
		\fi
		\expandafter\pgfplotstabletypeset@assign@final@colname\expandafter{\pgfplotstable@colname@out}\pgfplotstable@colname@out
		%
		\pgfplotstable@debug@notify@preprocess@col\pgfplotstable@colname@out
		%
		\expandafter\pgfplotslistpushbackglobal\pgfplotstable@colname@out\to\pgfplotstabletypeset@final@colnames
		\pgfkeysgetvalue{/pgfplots/table/column type}{\pgfplotstable@coltype}%
		\expandafter\pgfplotslistpushbackglobal\pgfplotstable@coltype\to\pgfplotstabletypeset@final@coltypes
		%
		\pgfplotslistnewempty\pgfplotstable@col@processed
		\c@pgfplotstable@counta=0 
		\pgfplotslistforeachungrouped\pgfplotstable@col\as\pgfplotstable@entry{%
			\pgfplotstableuserowtrue
			\edef\pgfplotstable@loc@TMPa{\noexpand\pgfkeysvalueof{/pgfplots/table/row predicate/.@cmd}\the\c@pgfplotstable@rowindex}%
			\pgfplotstable@loc@TMPa\pgfeov
			\ifpgfplotstableuserow
				\ifnum\pgfplotstable@firstnumrows=-1\relax
				\else
					\ifnum\c@pgfplotstable@counta<\pgfplotstable@firstnumrows\relax
					\else
						\t@pgfplots@toka=\expandafter{\pgfplotstable@entry}%
						\t@pgfplots@tokb=\expandafter{\pgfplotstable@colname@out}%
						\pgfplots@warning{Unbalanced cell with content '\the\t@pgfplots@toka' of column '\the\t@pgfplots@tokb' has been skipped: row count \the\c@pgfplotstable@counta+1 > \pgfplotstable@firstnumrows (which is the number of rows in the first column)}%
						\pgfplotstableuserowfalse
					\fi
				\fi
				\ifpgfplotstableuserow
					\expandafter\pgfplotstabletypeset@getfinalentry\expandafter{\pgfplotstable@entry}{\pgfplotstable@entry}%
					\expandafter\pgfplotslistpushback\pgfplotstable@entry\to\pgfplotstable@col@processed
					\advance\c@pgfplotstable@counta by1\relax
				\fi
			\fi
			\advance\c@pgfplotstable@rowindex by1\relax
		}%
		\ifnum\pgfplotstable@firstnumrows=-1\relax
			\xdef\pgfplotstable@firstnumrows{\the\c@pgfplotstable@counta}%
		\else
			% balance columns:
			\pgfplotstable@insertemptycells@forbalance
		\fi
		%
		\pgfplotstable@debug@notify@preprocessed@col\pgfplotstable@colname@out
		%
		\expandafter\pgfplotslistpushbackglobal\expandafter{\pgfplotstable@col@processed}\to\pgfplotstabletypeset@final@cols
		\global\advance\c@pgfplotstable@colindex by1\relax
	}%
	%
	\pgfplotstable@disable@column@styles
	%
	\pgfplotslistcheckempty\pgfplotstabletypeset@final@colnames
	\ifpgfplotslistempty
		\let\pgfplotstable@result=\pgfutil@empty
	\else
		% Ok, I have now everything which will come into the final table.
		%
		% But I have it column-oriented; I need to transpose the storage.
		%
		% The following code assembles a
		% \begin{tabular}{}
		% ...
		% \end{tabular}
		% statement piece after piece.
		%
	%\message{I have now \meaning\pgfplotstabletypeset@final@colnames, and \meaning\pgfplotstabletypeset@final@cols.}%
		% Step 1: column names.
		\c@pgfplotstable@colindex=0\relax
		\c@pgfplotstable@rowindex=-1\relax
		\let\pgfplotstable@result=\pgfutil@empty
		%
		\pgfkeysgetvalue{/pgfplots/table/font}{\pgfplotstable@font}%
		\ifx\pgfplotstable@font\pgfutil@empty
		\else
			\t@pgfplots@toka=\expandafter{\pgfplotstable@font}%
			\edef\pgfplotstable@curline{\noexpand\begingroup\the\t@pgfplots@toka}%
			\pgfplotstable@curline@FLUSH
		\fi
		%
		\pgfkeysgetvalue{/pgfplots/table/begin table}{\pgfplotstable@entry}%
		\t@pgfplots@toka=\expandafter{\pgfplotstable@entry}%
		\edef\pgfplotstable@curline{\the\t@pgfplots@toka}%
		%
		\ifpgfplotstabletypesetskipcoltypes
		\else
			% STEP 1.1: collect column types:
			\def\pgfplotstable@resulttypes{}%
			\pgfplotslistforeachungrouped\pgfplotstabletypeset@final@coltypes\as\pgfplotstable@coltype{%
				\t@pgfplots@toka=\expandafter{\pgfplotstable@resulttypes}%
				\t@pgfplots@tokb=\expandafter{\pgfplotstable@coltype}%
				\edef\pgfplotstable@resulttypes{\the\t@pgfplots@toka\the\t@pgfplots@tokb}%
			}%
			\ifx\pgfplotstable@resulttypes\pgfutil@empty
			\else
				\t@pgfplots@toka=\expandafter{\pgfplotstable@curline}%
				\t@pgfplots@tokb=\expandafter{\pgfplotstable@resulttypes}%
				\edef\pgfplotstable@curline{\the\t@pgfplots@toka{\the\t@pgfplots@tokb}}%
			\fi
		\fi
		\ifx\pgfplotstable@curline\pgfutil@empty
		\else
			\pgfplotstable@curline@FLUSH
		\fi
		%
		% Step 1.2: Collect FIRST ROW (column names)
		\begingroup
		\pgfplotstableset{every head row}%
		\pgfkeysgetvalue{/pgfplots/table/before row}{\pgfplotstable@before}%
		\pgfkeysgetvalue{/pgfplots/table/after row}{\pgfplotstable@after}%
		\t@pgfplots@toka=\expandafter{\pgfplotstable@before}%
		\t@pgfplots@tokb=\expandafter{\pgfplotstable@after}%
		\xdef\pgfplots@glob@TMPc{%
			\noexpand\def\noexpand\pgfplotstable@before{\the\t@pgfplots@toka}%
			\noexpand\def\noexpand\pgfplotstable@after{\the\t@pgfplots@tokb}%
		}%
		\pgfkeysgetvalue{/pgfplots/table/typeset cell/.@cmd}\pgfplots@loc@TMPa
		\global\let\pgfplots@glob@TMPd=\pgfplots@loc@TMPa
		\endgroup
		\pgfplots@glob@TMPc
		\let\pgfplotstable@headrow@typesetcell=\pgfplots@glob@TMPd
		% insert 'before row' here:
		\t@pgfplots@toka=\expandafter{\pgfplotstable@before}%
		\t@pgfplots@tokb=\expandafter{\pgfplotstable@curline}%
		\edef\pgfplotstable@curline{\the\t@pgfplots@tokb\the\t@pgfplots@toka}%
		%
		\def\pgfplotstablecolname{\pgfplotstable@colname@out}%
		\pgfplotslistforeachungrouped\pgfplotstabletypeset@final@colnames\as\pgfplotstable@colname@out{%
			%
			\advance\c@pgfplotstable@colindex by1\relax
			% typeset the cell:
			\expandafter\pgfplotstable@headrow@typesetcell\pgfplotstable@colname@out\pgfeov
			\pgfkeysgetvalue{/pgfplots/table/@cell content}\pgfmathresult
			%
			% append this cell:
			\t@pgfplots@toka=\expandafter{\pgfplotstable@curline}%
			\t@pgfplots@tokb=\expandafter{\pgfmathresult}%
			\edef\pgfplotstable@curline{\the\t@pgfplots@toka\the\t@pgfplots@tokb}%
		}%
		% insert 'after row' here:
		\t@pgfplots@toka=\expandafter{\pgfplotstable@curline}%
		\t@pgfplots@tokb=\expandafter{\pgfplotstable@after}%
		\edef\pgfplotstable@curline{\the\t@pgfplots@toka\the\t@pgfplots@tokb}%
		%
		\pgfplotstable@curline@FLUSH
		%
	%\message{I have now \meaning\pgfplotstable@result.}%
		% Step 2: column contents.
		% I will first convert \pgfplotstabletypeset@final@cols into an array.
		\c@pgfplotstable@colindex=0\relax
		\pgfplotsarraynewempty\pgfplotstabletypeset@final@cols@array
		\pgfplotslistforeachungrouped\pgfplotstabletypeset@final@cols\as\pgfplotstable@col@processed{%
			\expandafter\pgfplotsarraypushback\expandafter{\pgfplotstable@col@processed}\to\pgfplotstabletypeset@final@cols@array
		}%
		% init numrows:
		\pgfplotsarrayselect\c@pgfplotstable@colindex\of\pgfplotstabletypeset@final@cols@array\to\pgfplotstable@col@processed
		\pgfplotslistsize\pgfplotstable@col@processed\to\c@pgfplotstable@numrows
		%
		% Now, we loop over every column as long as there are still rows
		% left. We assemble rows while we go.
		%
		\c@pgfplotstable@rowindex=0\relax
		\ifnum\c@pgfplotstable@colindex<\c@pgfplotstable@numcols
			\pgfplots@loop@CONTINUEtrue
		\else
			\pgfplots@loop@CONTINUEfalse
		\fi
		\def\pgfplotstablecolname{\pgfplotstable@error{Sorry, you can't evaluate \string\pgfplotstablecolname\space in this context.}}%
		\pgfutil@loop
		\ifpgfplots@loop@CONTINUE
			\pgfplotsarrayselect\c@pgfplotstable@colindex\of\pgfplotstabletypeset@final@cols@array\to\pgfplotstable@col@processed
			\pgfplotslistcheckempty\pgfplotstable@col@processed
			\ifpgfplotslistempty
				% assume that each column has the same number of entries
				% (normalised tables):
				\pgfplots@loop@CONTINUEfalse
			\else
				\ifnum\c@pgfplotstable@colindex=0\relax
					% Install styles for the next row.
					\begingroup
					\ifodd\c@pgfplotstable@rowindex
						\t@pgfplots@toka={every odd row}%
					\else
						\t@pgfplots@toka={every even row}%
					\fi
					\ifnum\c@pgfplotstable@rowindex=0\relax
						\t@pgfplots@toka=\expandafter{\the\t@pgfplots@toka,every first row}%
					\fi
					\ifpgfplotstable@disable@rowcolstyles
					\else
						\edef\pgfplotstable@loc@TMPa{\the\t@pgfplots@toka,every row no \the\c@pgfplotstable@rowindex/.try}%
						\t@pgfplots@toka=\expandafter{\pgfplotstable@loc@TMPa}%
						%
						\ifnum\c@pgfplotstable@rowindex=0\relax
						\else
							% process 'every nth row' styles:
							\begingroup
							\pgfplots@each@nth@styles
							\pgfmath@smuggleone\pgfplotsretval
							\endgroup
							\ifx\pgfplotsretval\pgfutil@empty
							\else
								\t@pgfplots@tokb=\expandafter{\pgfplotsretval}%
								\edef\pgfplotstable@loc@TMPa{\the\t@pgfplots@toka,\the\t@pgfplots@tokb}%
								\t@pgfplots@toka=\expandafter{\pgfplotstable@loc@TMPa}%
							\fi
						\fi
					\fi
					% misuse as temporary variable:
					\c@pgfplotstable@colindex=\c@pgfplotstable@rowindex
					\advance\c@pgfplotstable@colindex by1\relax
					\ifnum\c@pgfplotstable@colindex=\c@pgfplotstable@numrows
						\t@pgfplots@toka=\expandafter{\the\t@pgfplots@toka,every last row}%
					\fi
					\expandafter\pgfplotstableset\expandafter{\the\t@pgfplots@toka}%
					\pgfkeysgetvalue{/pgfplots/table/before row}{\pgfplotstable@before}%
					\pgfkeysgetvalue{/pgfplots/table/after row}{\pgfplotstable@after}%
					\t@pgfplots@toka=\expandafter{\pgfplotstable@before}%
					\t@pgfplots@tokb=\expandafter{\pgfplotstable@after}%
					\xdef\pgfplots@glob@TMPc{%
						\noexpand\def\noexpand\pgfplotstable@before{\the\t@pgfplots@toka}%
						\noexpand\def\noexpand\pgfplotstable@after{\the\t@pgfplots@tokb}%
					}%
					\endgroup
					\pgfplots@glob@TMPc
					% insert 'before row' here:
					\t@pgfplots@toka=\expandafter{\pgfplotstable@before}%
					\t@pgfplots@tokb=\expandafter{\pgfplotstable@curline}%
					\edef\pgfplotstable@curline{\the\t@pgfplots@tokb\the\t@pgfplots@toka}%
				\fi
				%
				%
				\pgfplotslistpopfront\pgfplotstable@col@processed\to\pgfplotstable@entry
				\pgfplotsarrayletentry\c@pgfplotstable@colindex\of\pgfplotstabletypeset@final@cols@array=\pgfplotstable@col@processed
				\advance\c@pgfplotstable@colindex by1\relax
				% typeset the cell:
				\pgfkeysgetvalue{/pgfplots/table/typeset cell/.@cmd}\pgfplots@loc@TMPa
				\expandafter\pgfplots@loc@TMPa\pgfplotstable@entry\pgfeov
				\pgfkeysgetvalue{/pgfplots/table/@cell content}\pgfmathresult
				%
				% append this cell:
				\t@pgfplots@toka=\expandafter{\pgfplotstable@curline}%
				\t@pgfplots@tokb=\expandafter{\pgfmathresult}%
				\edef\pgfplotstable@curline{\the\t@pgfplots@toka\the\t@pgfplots@tokb}%
				%
				\ifnum\c@pgfplotstable@colindex=\c@pgfplotstable@numcols\relax
					\c@pgfplotstable@colindex=0\relax
					% insert 'after row' here:
					\t@pgfplots@toka=\expandafter{\pgfplotstable@curline}%
					\t@pgfplots@tokb=\expandafter{\pgfplotstable@after}%
					\edef\pgfplotstable@curline{\the\t@pgfplots@toka\the\t@pgfplots@tokb}%
					\advance\c@pgfplotstable@rowindex by1\relax
					\pgfplotstable@curline@FLUSH
				\fi
	%\message{I have now \meaning\pgfplotstable@result.}%
			\fi
		\pgfutil@repeat
		\t@pgfplots@toka=\expandafter{\pgfplotstable@curline}%
		\pgfkeysgetvalue{/pgfplots/table/end table}{\pgfplotstable@entry}%
		\t@pgfplots@tokb=\expandafter{\pgfplotstable@entry}%
		\edef\pgfplotstable@curline{\the\t@pgfplots@toka\the\t@pgfplots@tokb}%
		\ifx\pgfplotstable@curline\pgfutil@empty
		\else
			\pgfplotstable@curline@FLUSH
		\fi
		%
		\ifx\pgfplotstable@font\pgfutil@empty
		\else
			\edef\pgfplotstable@curline{\noexpand\endgroup}%
			\pgfplotstable@curline@FLUSH
		\fi
		\ifx\pgfplotstable@outfilename\pgfutil@empty
		\else
			\immediate\closeout\pgfplotstable@outfile
		\fi
	%\message{I have now \meaning\pgfplotstable@result.}%
		\def\pgfplotstablecol{\pgfplotstable@error{Sorry, you can't access the \string\pgfplotstablecol\ in this context. It is ONLY valid during the preparation routines (please check the 'display columns/<index>' style in the manual).}}%
		\def\pgfplotstablerow{\pgfplotstable@error{Sorry, you can't access the \string\pgfplotstablerow\ in this context. It is ONLY valid during the preparation routines (please check the 'every row no <index>' style).}}%
		\ifpgfplotstabletypesetresult
			\pgfplotstable@result
		\fi
	\fi
	\pgfkeysgetvalue{/pgfplots/table/write to macro}\pgfplots@loc@TMPa
	\ifx\pgfplots@loc@TMPa\pgfutil@empty
		\global\let\pgfplots@glob@TMPa=\relax
	\else
		\t@pgfplots@toka=\expandafter{\pgfplotstable@result}%
		\xdef\pgfplots@glob@TMPa{\noexpand\def\expandafter\noexpand\pgfplots@loc@TMPa{\the\t@pgfplots@toka}}%
	\fi
	\endgroup
	\pgfplots@glob@TMPa% execute 'write to macro' if set.
}%

\def\pgfplotstable@disable@column@styles@error#1{%
	\pgfplotsthrow{invalid argument}\pgfplots@loc@TMPa{Sorry, the key '#1' has been assigned while processing row options. However, it needs to be invoked while processing column options. If your options depend on specific row indices, consider using \string\pgfplotstablerow\space and \string\pgfplotstablerows}\pgfeov%
}%
\def\pgfplotstable@disable@column@styles@#1{%
	\pgfkeysdef{/pgfplots/table/#1/.code}{\pgfplotstable@disable@column@styles@error{#1}}%
	\pgfkeysdef{/pgfplots/table/#1/.append code}{\pgfplotstable@disable@column@styles@error{#1}}%
	\pgfkeysdefargs{/pgfplots/table/#1/.add code}{##1##2}{\pgfplotstable@disable@column@styles@error{#1}}%
	\pgfkeysdef{/pgfplots/table/#1/.append style}{\pgfplotstable@disable@column@styles@error{#1}}%
}
\def\pgfplotstable@disable@column@styles{%
	\pgfplotstable@disable@column@styles@{postproc cell content}%
	\pgfplotstable@disable@column@styles@{preproc cell content}%
	\pgfplotstable@disable@column@styles@{assign cell content}%
}%

\newif\ifpgfplotstable@isfirstrow
\newif\ifpgfplotstable@islastrow
\newif\ifpgfplotstablecreatecol@isreallynew

\def\pgfplotstablecreatecol@opt[#1]#2#3{%
	\begingroup
	\def\pgfplotstablename{#3}% the name of the table struct
	\pgfplotstableset{columns=,#1,%
		/pgf/fpu/handlers/empty number/.code 2 args={%
			\pgfmathfloatcreate{0}{0.0}{0}%
		}%
	}%
	\pgfkeysgetvalue{/pgfplots/table/columns}{\pgfplotstable@colnames}%
	\ifx\pgfplotstable@colnames\pgfutil@empty
		\pgfplotstablegetcolumnlist#3\to\pgfplotstable@colnames
	\else
		\expandafter\pgfplotslistnew\expandafter\pgfplotstable@colnames\expandafter{\pgfplotstable@colnames}%
	\fi
	\pgfplotslistnewempty\pgfplotstable@colnames@real
	\t@pgfplots@toka=\expandafter{#2}% this should handle macro names in '#2'.
	\edef\pgfplotstable@newcolname{\the\t@pgfplots@toka}%
	\global\pgfplotstablecreatecol@isreallynewtrue
	\c@pgf@countd=0
	\pgfplotslistforeachungrouped\pgfplotstable@colnames\as\pgfplotstable@colname{%
		\pgfplotstable@is@colname\pgfplotstable@colname
		\ifpgfplotstableread@foundcolnames
			\expandafter\pgfplotslistpushback\expandafter{\pgfplotstable@colname}\to\pgfplotstable@colnames@real
		\else
			\pgfplotstablegetcolumnnamebyindex\pgfplotstable@colname\of#2\to\pgfplotstable@colname
			\expandafter\pgfplotslistpushback\expandafter{\pgfplotstable@colname}\to\pgfplotstable@colnames@real
		\fi
		\ifx\pgfplotstable@colname\pgfplotstable@newcolname
			\global\pgfplotstablecreatecol@isreallynewfalse
		\fi
		\expandafter\edef\csname pgfplotstablecreate@index@to@name@\the\c@pgf@countd\endcsname{\pgfplotstable@colname}%
		\advance\c@pgf@countd by1
	}%
	\edef\pgfplotstablecols{\the\c@pgf@countd}%
	\pgfplotstable@isfirstrowtrue
	\pgfplotstable@islastrowfalse
	\pgfplotslistnewempty\pgfplotstable@newcol
	%
	\let\c@pgfplotstable@numrows=\c@pgf@countd
	\let\c@pgfplotstable@rowindex=\c@pgf@countc
	\c@pgfplotstable@numrows=-1\relax
	\c@pgfplotstable@rowindex=0\relax
	\def\pgfplotstablerow{\the\c@pgfplotstable@rowindex}%
	\def\pgfplotstablerows{\the\c@pgfplotstable@numrows}%
	%
	\def\prevrow##1{\pgfplotstable@thisrow@impl{##1}{pgfplotstablecreate@prev@}{\pgfplotstable@thisrow@impl@}}%
	\def\thisrow##1{\pgfplotstable@thisrow@impl{##1}{pgfplotstablecreate@cur@}{\pgfplotstable@thisrow@impl@}}%
	\def\nextrow##1{\pgfplotstable@thisrow@impl{##1}{pgfplotstablecreate@next@}{\pgfplotstable@thisrow@impl@}}%
	\def\pgfplotstable@rowno@impl##1##2{%
		\pgfutil@ifundefined{pgfplotstablecreate@index@to@name@##1}{%
			\csname ##2row\endcsname{colindex##1}%
		}{%
			\csname ##2row\endcsname{\csname pgfplotstablecreate@index@to@name@##1\endcsname}%
		}%
	}%
	\def\prevrowno##1{\pgfplotstable@rowno@impl{##1}{prev}}%
	\def\thisrowno##1{\pgfplotstable@rowno@impl{##1}{this}}%
	\def\nextrowno##1{\pgfplotstable@rowno@impl{##1}{next}}%
	%
	\def\getprevrow##1##2{\expandafter\let\expandafter##2\csname pgfplotstablecreate@prev@##1\endcsname}%
	\def\getthisrow##1##2{\expandafter\let\expandafter##2\csname pgfplotstablecreate@cur@##1\endcsname}%
	\def\getnextrow##1##2{\expandafter\let\expandafter##2\csname pgfplotstablecreate@next@##1\endcsname}%
	%
	\pgfplots@loop@CONTINUEtrue
	% allow this here to accumulate something.
	\pgfutil@ifundefined{pgfmathaccuma}{%
		\let\pgfmathaccuma=\pgfutil@empty
	}{}%
	\pgfutil@ifundefined{pgfmathaccumb}{%
		\let\pgfmathaccumb=\pgfutil@empty
	}{}%
	\pgfutil@loop% loop over each row until there are no more rows.
	\ifpgfplots@loop@CONTINUE
		\ifnum\c@pgfplotstable@numrows=-1\relax
			\pgfplotslistfront\pgfplotstable@colnames\to\pgfplotstable@colname
			\expandafter\pgfplotslistsize\csname\string#3@\pgfplotstable@colname\endcsname\to\c@pgfplotstable@numrows
		\fi
		\pgfplotslistforeachungrouped\pgfplotstable@colnames\as\pgfplotstable@colname{%
			\expandafter\pgfplotstableresolvecolname\expandafter{\pgfplotstable@colname}\of#3\to\pgfplotstable@real@colname
			\ifpgfplotstable@isfirstrow
				\expandafter\pgfplotslistcheckempty\csname\string#3@\pgfplotstable@real@colname\endcsname
				\ifpgfplotslistempty
					% the table is completely empty. break.
					\pgfplots@loop@CONTINUEfalse
				\else
					\expandafter\let\csname pgfplotstablecreate@prev@\pgfplotstable@colname\endcsname=\pgfutil@empty
					\expandafter\pgfplotslistpopfront\csname\string#3@\pgfplotstable@real@colname\endcsname\to\pgfplotstable@entry
					\expandafter\let\csname pgfplotstablecreate@cur@\pgfplotstable@colname\endcsname=\pgfplotstable@entry
				\fi
			\else
				\expandafter\let\expandafter\pgfplotstable@prev\csname pgfplotstablecreate@cur@\pgfplotstable@colname\endcsname
				\expandafter\let\csname pgfplotstablecreate@prev@\pgfplotstable@colname\endcsname=\pgfplotstable@prev
				%
				\expandafter\let\expandafter\pgfplotstable@next\csname pgfplotstablecreate@next@\pgfplotstable@colname\endcsname
				\expandafter\let\csname pgfplotstablecreate@cur@\pgfplotstable@colname\endcsname=\pgfplotstable@next
			\fi
			\expandafter\pgfplotslistcheckempty\csname\string#3@\pgfplotstable@real@colname\endcsname
			\ifpgfplotslistempty
				\expandafter\let\csname pgfplotstablecreate@next@\pgfplotstable@colname\endcsname=\pgfutil@empty
				\pgfplotstable@islastrowtrue
			\else
				\expandafter\pgfplotslistpopfront\csname\string#3@\pgfplotstable@real@colname\endcsname\to\pgfplotstable@entry
				\expandafter\let\csname pgfplotstablecreate@next@\pgfplotstable@colname\endcsname=\pgfplotstable@entry
			\fi
		}%
		\ifpgfplots@loop@CONTINUE
			% Compute content:
			%
			\begingroup
			\ifpgfplotstable@isfirstrow
				\pgfkeysvalueof{/pgfplots/table/create col/assign first/.@cmd}\pgfeov
			\else
				\ifpgfplotstable@islastrow
					\pgfkeysvalueof{/pgfplots/table/create col/assign last/.@cmd}\pgfeov
				\else
					\pgfkeysvalueof{/pgfplots/table/create col/assign/.@cmd}\pgfeov
				\fi
			\fi
			\pgfkeysgetvalue{/pgfplots/table/create col/next content}{\pgfplotstable@entry}%
			\t@pgfplots@toka=\expandafter{\pgfplotstable@entry}%
			\t@pgfplots@tokb=\expandafter{\pgfmathaccuma}%
			\xdef\pgfplots@glob@TMPc{%
				\noexpand\def\noexpand\pgfplotstable@entry{\the\t@pgfplots@toka}%
				\noexpand\def\noexpand\pgfmathaccuma{\the\t@pgfplots@tokb}%
			}%
			\pgfmath@smuggleone\pgfmathaccumb
			\endgroup
			\pgfplots@glob@TMPc
			\expandafter\pgfplotslistpushback\expandafter{\pgfplotstable@entry}\to\pgfplotstable@newcol
			%
			\ifpgfplotstable@islastrow
				\pgfplots@loop@CONTINUEfalse
			\fi
		\fi
		\pgfplotstable@isfirstrowfalse
		\advance\c@pgfplotstable@rowindex by1\relax
	\pgfutil@repeat
	\global\let\pgfplots@glob@TMPc=\pgfplotstable@newcol
	\global\let\pgfplots@glob@TMPb=\pgfplotstable@newcolname
	\endgroup
	\ifpgfplotstablecreatecol@isreallynew
		\expandafter\pgfplotslistpushback\expandafter{\pgfplots@glob@TMPb}\to#3%
	\fi
	\expandafter\let\csname\string#3@\pgfplots@glob@TMPb\endcsname=\pgfplots@glob@TMPc
}%


%-----------------------------------------------------------
%
% Implementation of '/pgfplots/table/create col/function graph cut y':
%
%-----------------------------------------------------------

%	create on use/cut/.style={create col/function graph cut y={7e-4}{x=Basis,ymode=log,xmode=log}{{table=regtable,y=special-L2}}},
% #1 = value of fixed line (I call it epsilon)
% #2 = options
% #3 = specification where to get the y from. It is a comma separated
% list of key-value sets, on set for every y(x) graph which shall be
% used.
% #4 = the direction in which the line does NOT vary (either 'x' or 'y')
% #5 = the direction in which the line DOES vary (either 'x' or 'y')
\def\pgfplotstable@fgc@init#1#2#3#4#5{%
	\def\pgfplotstable@fgc@xmode{0}%
	\def\pgfplotstable@fgc@ymode{0}%
	%
	\pgfkeysalso{/pgf/fpu=true}%
	%
	\pgfqkeys{/pgfplots/table/create col/function graph cut}{#2}%
	\pgfkeysgetvalue{/pgfplots/table/create col/function graph cut/table}\pgfplotstable@fgc@table
	\pgfkeysgetvalue{/pgfplots/table/create col/function graph cut/x}\pgfplotstable@fgc@x
	\pgfkeysgetvalue{/pgfplots/table/create col/function graph cut/y}\pgfplotstable@fgc@y
	\ifx\pgfplotstable@fgc@table\pgfutil@empty
		% no table here, ok.
	\else
		% ok, we need to query (or load) a table!
		\expandafter\pgfplotstable@isloadedtable\expandafter{\pgfplotstable@fgc@table}{%
			\expandafter\let\expandafter\pgfplotstable@fgc@table\pgfplotstable@fgc@table
		}{%
			\expandafter\pgfplotstableread\expandafter{\pgfplotstable@fgc@table}{\pgfplotstable@fgc@table}%
		}%
	\fi
	%
	\ifnum\csname pgfplotstable@fgc@#4mode\endcsname=1
		\pgfmathparse{ln(#1)}%
	\else
		\pgfmathparse{#1}%
	\fi
	\let\pgfplotstable@fgc@eps=\pgfmathresult
	\pgfplots@assign@list\pgfmathaccumb{#3}%
	\pgfkeysgetvalue{/pgfplots/table/create col/function graph cut/foreach}\pgfplotstable@fgc@foreach@
	\ifx\pgfplotstable@fgc@foreach@\pgfutil@empty
	\else
		\edef\pgfplotstable@fgc@foreach@process@append{x={\pgfplotstable@fgc@x},y={\pgfplotstable@fgc@y}}%
		% we have something like 
		%   foreach={\d in {1,2,3,4}}{table\d}
		% -> process it!
		%  This will modify \pgfmathaccumb
		\expandafter\pgfplotstable@fgc@foreach@process\pgfplotstable@fgc@foreach@\pgfplots@EOI
		\pgfkeyssetvalue{/pgfplots/table/create col/function graph cut/foreach}{}%
	\fi
	%
	\pgfkeysdef{/pgfplots/table/create col/assign}{%
%\message{working on next cell ...}%
		\pgfplotslistcheckempty\pgfmathaccumb
		\ifpgfplotslistempty
			\pgfkeyslet{/pgfplots/table/create col/next content}\pgfutil@empty%
		\else
			\pgfplotslistpopfront\pgfmathaccumb\to\pgfmathresult
%\message{working on next cell: \meaning\pgfmathresult ...}%
			\global\let\pgfplotstable@fgc@foreach=\pgfutil@empty
			\begingroup
				% set local keys:
				\def\pgfplots@loc@TMPa{\pgfqkeys{/pgfplots/table/create col/function graph cut}}%
				\expandafter\pgfplots@loc@TMPa\expandafter{\pgfmathresult}%
				% acquire any sources:
				\pgfkeysgetvalue{/pgfplots/table/create col/function graph cut/table}\pgfplotstable@fgc@table
				\pgfkeysgetvalue{/pgfplots/table/create col/function graph cut/x}\pgfplotstable@fgc@x
				\pgfkeysgetvalue{/pgfplots/table/create col/function graph cut/y}\pgfplotstable@fgc@y
				\pgfkeysgetvalue{/pgfplots/table/create col/function graph cut/foreach}\pgfplotstable@fgc@foreach@
				\ifx\pgfplotstable@fgc@foreach@\pgfutil@empty
					\ifx\pgfplotstable@fgc@table\pgfutil@empty
					\else
						% ok, we need to query (or load) a table!
						\expandafter\pgfplotstable@isloadedtable\expandafter{\pgfplotstable@fgc@table}{%
							\expandafter\let\expandafter\pgfplotstable@fgc@table\pgfplotstable@fgc@table
						}{%
							\expandafter\pgfplotstableread\expandafter{\pgfplotstable@fgc@table}{\pgfplotstable@fgc@table}%
						}%
					\fi
					\expandafter\pgfplotstablegetcolumnbyname\expandafter{\pgfplotstable@fgc@x}\of\pgfplotstable@fgc@table\to\pgfplotstable@fgc@x@col
					\expandafter\pgfplotstablegetcolumnbyname\expandafter{\pgfplotstable@fgc@y}\of\pgfplotstable@fgc@table\to\pgfplotstable@fgc@y@col
					%
					% search for epsilon in the y values:
					\expandafter\pgfplotstable@fgc@findintervalwitheps\expandafter{\csname pgfplotstable@fgc@#4@col\endcsname}{\csname pgfplotstable@fgc@#4mode\endcsname}%
					\ifx\pgfplotstable@fgc@second\pgfutil@empty
						\let\pgfmathresult=\pgfutil@empty
					\else
						%
						% acquire the associated x values:
						\expandafter\pgfplotstable@fgc@getassociated\expandafter{\csname pgfplotstable@fgc@#5@col\endcsname}{\csname pgfplotstable@fgc@#5mode\endcsname}%
						%
						% Interpolate:
						\pgfmathparse{
							\pgfplotstable@fgc@abscissafirst + 
							(\pgfplotstable@fgc@eps - \pgfplotstable@fgc@first) / (\pgfplotstable@fgc@second-\pgfplotstable@fgc@first) 
							* (\pgfplotstable@fgc@abscissasecond - \pgfplotstable@fgc@abscissafirst) }%
						\if\pgfplotstable@fgc@xmode1
							\pgfmathfloatexp@{\pgfmathresult}%
						\fi
						\pgflibraryfpuifactive{\pgfmathfloattosci{\pgfmathresult}}{}%
%\message{FGC: found \pgfplotstable@fgc@eps\space in no \pgfplotstable@fgc@index; \pgfplotstable@fgc@first:\pgfplotstable@fgc@second; lastwasless=\pgfplotstable@fgc@lastwasless; abscissa interval = \pgfplotstable@fgc@abscissafirst:\pgfplotstable@fgc@abscissasecond; result = \pgfmathresult}%
					\fi
				\else
					\global\let\pgfplotstable@fgc@foreach=\pgfplotstable@fgc@foreach@
					\edef\pgfmathresult{x={\pgfplotstable@fgc@x},y={\pgfplotstable@fgc@y}}%
				\fi
				%
				\pgfmath@smuggleone\pgfmathresult
			\endgroup
			\ifx\pgfplotstable@fgc@foreach\pgfutil@empty
				\pgfkeyslet{/pgfplots/table/create col/next content}\pgfmathresult%
			\else
				% we have something like 
				%   foreach={\d in {1,2,3,4}}{table\d}
				% -> process it!
				%  This will modify \pgfmathaccumb
				\let\pgfplotstable@fgc@foreach@process@append=\pgfmathresult
				\expandafter\pgfplotstable@fgc@foreach@process\pgfplotstable@fgc@foreach\pgfplots@EOI
			\fi
		\fi
	}%
}%

\def\pgfplotstable@fgc@foreach@process{%
	\pgfutil@ifnextchar\bgroup
		{\pgfplotstable@fgc@foreach@process@a}{%
		\pgfplotstable@fgc@foreach@process@error}%
}%
\def\pgfplotstable@fgc@foreach@process@a#1{%
	\pgfutil@ifnextchar\bgroup
		{\pgfplotstable@fgc@foreach@process@b{#1}}{%
		\pgfplotstable@fgc@foreach@process@error}%
}%
\def\pgfplotstable@fgc@foreach@process@error#1\pgfplots@EOI{%
	\def\pgfplots@loc@TMPa{#1}%
	\pgfplots@command@to@string\pgfplots@loc@TMPa\pgfplots@loc@TMPb
	\pgfplotsthrow{invalid argument}{\pgfplots@loc@TMPa}{Sorry, I expected two arguments for /pgfplots/table/create col/function graph cut/foreach, but I found 'foreach=\pgfplots@loc@TMPb'. Perhaps the braces are not as expected?}\pgfeov%
}%
\def\pgfplotstable@fgc@foreach@process@b#1#2#3\pgfplots@EOI{%
	\global\pgfplotslistnewempty\pgfplots@glob@TMPa
	\begingroup
	\foreach #1 {%
		\edef\pgfplots@loc@TMPa{#2}%
		\expandafter\pgfplotslistpushfrontglobal\pgfplots@loc@TMPa\to\pgfplots@glob@TMPa
	}%
	\endgroup
	\pgfplotslistforeachungrouped\pgfplots@glob@TMPa\as\pgfplots@loc@TMPa{%
		\edef\pgfplots@loc@TMPa{table={\pgfplots@loc@TMPa},\pgfplotstable@fgc@foreach@process@append}%
		\expandafter\pgfplotslistpushfront\pgfplots@loc@TMPa\to\pgfmathaccumb
	}%
}%

% #1 : the coordinate list to query
% #2 : the mode
%
% POSTCONDITION:
% 	- \pgfplotstable@fgc@abscissafirst and	\pgfplotstable@fgc@abscissasecond
% 	contain the abscissa values
%
\def\pgfplotstable@fgc@getassociated#1#2{%
	\c@pgfplotstable@counta=\pgfplotstable@fgc@index
	\pgfplotslistselect\c@pgfplotstable@counta\of#1\to\pgfplotstable@fgc@abscissasecond
	\advance\c@pgfplotstable@counta by-1
	\pgfplotslistselect\c@pgfplotstable@counta\of#1\to\pgfplotstable@fgc@abscissafirst
	\pgfmathfloatparsenumber{\pgfplotstable@fgc@abscissafirst}%
	\let\pgfplotstable@fgc@abscissafirst=\pgfmathresult
	\pgfmathfloatparsenumber{\pgfplotstable@fgc@abscissasecond}%
	\let\pgfplotstable@fgc@abscissasecond=\pgfmathresult
	\if1#2
		\pgfmathln@{\pgfplotstable@fgc@abscissafirst}%
		\let\pgfplotstable@fgc@abscissafirst=\pgfmathresult
		\pgfmathln@{\pgfplotstable@fgc@abscissasecond}%
		\let\pgfplotstable@fgc@abscissasecond=\pgfmathresult
	\fi
}%

% Search for the (first) interval in #1 containing epsilon and return
% \pgfplotstable@fgc@first and \pgfplotstable@fgc@second.
%
% #1 : the list of values.
% #2 : the mode (0 = linear, 1 = log)
%
% PRECONDITION:
% 	- \pgfplotstable@fgc@eps contains the value to search for
%
% POSTCONDITION
% 	On success,
% 	- \pgfplotstable@fgc@first and \pgfplotstable@fgc@second contain the interval.
% 	- \pgfplotstable@fgc@index contains the index of the @second coordinate.
% 	- \pgfplotstable@fgc@lastwasless is
% 	     1 if \pgfplotstable@fgc@second < eps
% 	     0 if \pgfplotstable@fgc@second >= eps
% 	If there was no such interval,
% 	\pgfplotstable@fgc@second will be empty.
\def\pgfplotstable@fgc@findintervalwitheps#1#2{%
	\global\let\pgfplotstable@fgc@first=\pgfutil@empty
	\global\let\pgfplotstable@fgc@second=\pgfutil@empty
	\global\def\pgfplotstable@fgc@lastwasless{2}%
	\global\def\pgfplotstable@fgc@break{0}%
	\global\def\pgfplotstable@fgc@index{0}%
	\pgfplotslistforeach#1\as\pgfplotstable@fgc@val{%
		\if\pgfplotstable@fgc@break0
			\pgfmathfloatparsenumber{\pgfplotstable@fgc@val}%
			\let\pgfplotstable@fgc@val=\pgfmathresult
			\ifnum#2=1
				\pgfmathln@{\pgfplotstable@fgc@val}%
				\let\pgfplotstable@fgc@val=\pgfmathresult
			\fi
			\pgfmathfloatlessthan@{\pgfplotstable@fgc@val}{\pgfplotstable@fgc@eps}%
			\ifpgfmathfloatcomparison
				\pgfplotstable@fgc@findintervalwitheps@updateresult 01
			\else
				\pgfplotstable@fgc@findintervalwitheps@updateresult 10
			\fi
		\fi
	}%
}

% #1 : the value of \pgfplotstable@fgc@lastwasless for which the loop
% shall break.
% #2 : the next value for \pgfplotstable@fgc@lastwasless
\def\pgfplotstable@fgc@findintervalwitheps@updateresult#1#2{%
	\if\pgfplotstable@fgc@lastwasless#1
		% found it !
		\global\let\pgfplotstable@fgc@second=\pgfplotstable@fgc@val
		\global\def\pgfplotstable@fgc@break{1}%
	\else
		\global\let\pgfplotstable@fgc@first=\pgfplotstable@fgc@val
		\c@pgfplotstable@counta=\pgfplotstable@fgc@index
		\advance\c@pgfplotstable@counta by1
		\xdef\pgfplotstable@fgc@index{\the\c@pgfplotstable@counta}%
	\fi
	\global\def\pgfplotstable@fgc@lastwasless{#2}%
}%



\def\pgfplotstabletranspose@[#1]#2#3{%
	%\begingroup is already in \pgfplotstabletranspose
		\def\pgfplotstable@loc@TMPa{#1}%
		\ifx\pgfplotstable@loc@TMPa\pgfutil@empty
		\else
			\pgfplotstableset{#1}%
		\fi
		%
		\pgfkeysgetvalue{/pgfplots/table/colnames from}\pgfplotstabletranspose@outcolnames
		\pgfkeysgetvalue{/pgfplots/table/input colnames to}\pgfplotstabletranspose@incolnames
		\pgfkeysgetvalue{/pgfplots/table/columns}\pgfplotstable@colnames
		%
		\ifx\pgfplotstable@colnames\pgfutil@empty
			\pgfplotstableforeachcolumn{#3}\as\pgfplots@colname{%
				\t@pgfplots@toka=\expandafter{\pgfplots@colname}%
				\ifnum\pgfplotstablecol=0
					\edef\pgfplotstable@colnames{{\the\t@pgfplots@toka}}%
				\else
					\t@pgfplots@tokb=\expandafter{\pgfplotstable@colnames}%
					\edef\pgfplotstable@colnames{\the\t@pgfplots@tokb,{\the\t@pgfplots@toka}}%
				\fi
			}%
		\fi
		%
		% sanity checking:
		\ifx\pgfplotstabletranspose@outcolnames\pgfutil@empty
		\else
			\def\pgfplotstabletranspose@outcolnames@foundit{0}%
			\expandafter\pgfplotsutilforeachcommasep\expandafter{\pgfplotstable@colnames}\as\pgfplots@colname{%
				\ifx\pgfplots@colname\pgfplotstabletranspose@outcolnames
					\def\pgfplotstabletranspose@outcolnames@foundit{1}%
				\fi
			}%
			\if1\pgfplotstabletranspose@outcolnames@foundit
			\else
				% insert the 'colnames from' column into 'columns':
				\pgfplots@warning{table transposition: the 'colnames from=\pgfplotstabletranspose@outcolnames' is not part of 'columns'. Adding it.}%
				\t@pgfplots@toka=\expandafter{\pgfplotstabletranspose@outcolnames}%
				\t@pgfplots@tokb=\expandafter{\pgfplotstable@colnames}%
				\edef\pgfplotstable@colnames{\the\t@pgfplots@tokb,{\the\t@pgfplots@toka}}%
			\fi
		\fi
		%
		% The column NAMES are collected into this list:
		\global\pgfplotslistnewempty\pgfplotstable@colnames@glob
		% this thing counts output columns:
		\let\c@pgfplotstable@numoutcols=\c@pgf@countd
		\c@pgfplotstable@numoutcols=0
		\let\c@pgfplotstable@rowindex=\c@pgf@counta
		\c@pgfplotstable@rowindex=0
		%
		\ifx\pgfplotstabletranspose@incolnames\pgfutil@empty
		\else
			\expandafter\pgfplotslistpushbackglobal\expandafter{\pgfplotstabletranspose@incolnames}\to\pgfplotstable@colnames@glob
			\def\pgfplots@loc@TMPa{\pgfplotsapplistXnewempty[to global]}%
			\expandafter\pgfplots@loc@TMPa\csname pgfp@numtable@glob@col@0\endcsname
			\advance\c@pgfplotstable@numoutcols by1
		\fi
		\def\pgfplotstable@isfirstcol{1}%
		\expandafter\pgfplotsutilforeachcommasep\expandafter{\pgfplotstable@colnames}\as\pgfplots@colname{%
			\c@pgfplotstable@rowindex=0
			\ifx\pgfplotstabletranspose@incolnames\pgfutil@empty
			\else
				\def\pgfplotstabletranspose@useit{1}%
				\ifx\pgfplotstabletranspose@outcolnames\pgfutil@empty
				\else
					\ifx\pgfplotstabletranspose@outcolnames\pgfplots@colname
						\def\pgfplotstabletranspose@useit{0}%
					\fi
				\fi
				\if1\pgfplotstabletranspose@useit
					\expandafter\pgfplotslist@assembleentry\expandafter{\pgfplots@colname}\into\t@pgfplots@tokc
					\def\pgfplotstableread@TMP{\expandafter\pgfplotsapplistXpushback\expandafter{\the\t@pgfplots@tokc}\to}%
					\expandafter\pgfplotstableread@TMP\csname pgfp@numtable@glob@col@\the\c@pgfplotstable@rowindex\endcsname
				\fi
				\advance\c@pgfplotstable@rowindex by1
			\fi
			%
			\pgfplotstableforeachcolumnelement\pgfplots@colname\of#3\as\pgfplots@cell{%
				\if1\pgfplotstable@isfirstcol
					% prepare a new column for the output:
					\def\pgfplots@loc@TMPa{\pgfplotsapplistXnewempty[to global]}%
					\expandafter\pgfplots@loc@TMPa\csname pgfp@numtable@glob@col@\the\c@pgfplotstable@rowindex\endcsname
					%
					\ifx\pgfplotstabletranspose@outcolnames\pgfutil@empty
						\edef\pgfplotstable@loc@TMPa{\pgfplotstablerow}%
						\expandafter\pgfplotslistpushbackglobal\expandafter{\pgfplotstable@loc@TMPa}\to\pgfplotstable@colnames@glob
					\fi
					\advance\c@pgfplotstable@numoutcols by1
				\fi
				\def\pgfplotstabletranspose@useit{1}%
				\ifx\pgfplotstabletranspose@outcolnames\pgfutil@empty
				\else
					\ifx\pgfplotstabletranspose@outcolnames\pgfplots@colname
						\pgfutil@ifundefined{pgfplotstabletranspose@@\pgfplots@cell}{%
						}{%
							\let\pgfplots@loc@TMPa=\pgfplots@cell
							\pgfplotsthrow{non unique colname}{\pgfplots@loc@TMPa}{Sorry, 'colnames from=\pgfplotstabletranspose@outcolnames' doesn't yield unique column names (`\pgfplots@cell' comes twice)}\pgfeov%
							\ifx\pgfplots@loc@TMPa\pgfutil@empty
								\edef\pgfplots@cell{row no \pgfplotstablerow}%
							\fi
						}%
						\expandafter\pgfplotslistpushbackglobal\expandafter{\pgfplots@cell}\to\pgfplotstable@colnames@glob
						\expandafter\def\csname pgfplotstabletranspose@@\pgfplots@cell\endcsname{1}%
						\def\pgfplotstabletranspose@useit{0}%
					\fi
				\fi
				\if1\pgfplotstabletranspose@useit
					% append everything to the output column with index
					% \pgfplotstablerow
					\expandafter\pgfplotslist@assembleentry\expandafter{\pgfplots@cell}\into\t@pgfplots@tokc
					\def\pgfplotstableread@TMP{\expandafter\pgfplotsapplistXpushback\expandafter{\the\t@pgfplots@tokc}\to}%
					\expandafter\pgfplotstableread@TMP\csname pgfp@numtable@glob@col@\the\c@pgfplotstable@rowindex\endcsname
				\fi
				\advance\c@pgfplotstable@rowindex by1
			}%
			\def\pgfplotstable@isfirstcol{0}%
		}%
		%
		% flush the buffers of the applistX things:
		\c@pgfplotstable@rowindex=0
		\pgfutil@loop
		\ifnum\c@pgfplotstable@rowindex<\c@pgfplotstable@numoutcols
			\expandafter\pgfplotsapplistXflushbuffers\csname pgfp@numtable@glob@col@\the\c@pgfplotstable@rowindex\endcsname
			\advance\c@pgfplotstable@rowindex by1
		\pgfutil@repeat
		% finalize the global buffers:
		\def\pgfplotsscanlinelength{-1}%
		\edef\pgfplots@loc@TMPa{\pgfplotstablenameof{#3}_transposed}%
		\expandafter\pgfplotstable@copy@to@globalbuffers@simple\expandafter{\pgfplots@loc@TMPa}%
	\endgroup
	\pgfplotstable@copy@globalbuffers@to{#2}%
}%

% #1: keys
\def\pgfplotstable@linear@regression#1{%
	\begingroup
	\pgfqkeys{/pgfplots/table/create col/linear regression}{/pgf/fpu,#1}%
	\pgfkeysgetvalue{/pgfplots/table/create col/linear regression/x}{\pgfplotstable@xsrc}%
	\pgfkeysgetvalue{/pgfplots/table/create col/linear regression/y}{\pgfplotstable@ysrc}%
	\pgfkeysgetvalue{/pgfplots/table/create col/linear regression/table}{\pgfplotstable@table}%
	\pgfkeysgetvalue{/pgfplots/table/create col/linear regression/xmode}{\pgfplotstable@xmode}%
	\pgfkeysgetvalue{/pgfplots/table/create col/linear regression/ymode}{\pgfplotstable@ymode}%
	\pgfkeysgetvalue{/pgfplots/table/create col/linear regression/variance}{\pgfplotstable@variance@colname}%
	\pgfkeysgetvalue{/pgfplots/table/create col/linear regression/variance list}{\pgfplotstable@variance@list}%
	\pgfkeysgetvalue{/pgfplots/table/create col/linear regression/variance src}{\pgfplotstable@variance@table}%
	%
	\ifx\pgfplotstable@table\pgfutil@empty
		\pgfutil@ifundefined{pgfplotstablename}{}{% query the name of the actual table struct
			\let\pgfplotstable@table=\pgfplotstablename
		}%
	\fi
	\ifx\pgfplotstable@table\pgfutil@empty
		\pgfplots@error{Sorry, I couldn't determine a value for create col/linear regression/table. Which table should I load?}%
	\fi
	\ifx\pgfplotstable@xsrc\pgfutil@empty
		\pgfplotsifinaddplottablestruct{%
			\pgfutil@ifundefined{pgfplots@plot@tbl@x}{}{%
				\let\pgfplotstable@xsrc=\pgfplots@plot@tbl@x
				\ifx\pgfplotstable@ysrc\pgfutil@empty
					\pgfplotstablegetcolsof\pgfplots@table
					\ifnum\pgfplotsretval=2
					\else
						\pgfplotsthrow{invalid argument}{\pgfplotstable@ysrc}{Sorry, I don't which column should be used as `y' for the linear regression. Please provide 'linear regression={y=<colname>}'}\pgfeov%
					\fi
				\fi
			}%
		}{}%
	\fi
	\ifx\pgfplotstable@xsrc\pgfutil@empty
		\def\pgfplotstable@xsrc{[index]0}%
	\fi
	\ifx\pgfplotstable@ysrc\pgfutil@empty
		\def\pgfplotstable@ysrc{[index]1}%
	\fi
	%
	\t@pgfplots@toka=\expandafter{\pgfplotstable@table}%
	\t@pgfplots@tokb=\expandafter{\pgfplotstable@xsrc}%
	\t@pgfplots@tokc=\expandafter{\pgfplotstable@ysrc}%
	\edef\pgfplots@loc@TMPa{{\the\t@pgfplots@tokb}\noexpand\of{\the\t@pgfplots@toka}}%
	\edef\pgfplots@loc@TMPb{{\the\t@pgfplots@tokc}\noexpand\of{\the\t@pgfplots@toka}}%
	\expandafter\pgfplotstablegetcolumn\pgfplots@loc@TMPa\to\pgfplotstable@X
	\expandafter\pgfplotstablegetcolumn\pgfplots@loc@TMPb\to\pgfplotstable@Y
	%
	\edef\pgfplotstable@xmode{\pgfplotstable@xmode}%
	\expandafter\pgfplotstable@linear@regression@prepare@mode\expandafter{\pgfplotstable@xmode}{x}%%
	\edef\pgfplotstable@ymode{\pgfplotstable@ymode}%
	\expandafter\pgfplotstable@linear@regression@prepare@mode\expandafter{\pgfplotstable@ymode}{y}%%
	%
	\ifx\pgfplotstable@variance@list\pgfutil@empty
		% check 'variance' key (loaded from table)
		\pgfplotslistnewempty\pgfplotstable@VARIANCE
		\ifx\pgfplotstable@variance@colname\pgfutil@empty
		\else
			\ifx\pgfplotstable@variance@table\pgfutil@empty
				\t@pgfplots@toka=\expandafter{\pgfplotstable@table}%
				\t@pgfplots@tokb=\expandafter{\pgfplotstable@variance@colname}%
				\edef\pgfplots@loc@TMPa{{\the\t@pgfplots@tokb}\noexpand\of{\the\t@pgfplots@toka}}%
				\expandafter\pgfplotstablegetcolumn\pgfplots@loc@TMPa\to\pgfplotstable@VARIANCE
			\else
				\t@pgfplots@toka=\expandafter{\pgfplotstable@variance@colname}%
				\t@pgfplots@tokb=\expandafter{\pgfplotstable@variance@table}%
				\edef\pgfplotstable@loc@TMPa{%
					\noexpand\pgfplotstablegetcolumn{\the\t@pgfplots@toka}\noexpand\of{\the\t@pgfplots@tokb}\noexpand\to\noexpand\pgfplotstable@VARIANCE}%
				\pgfplotstable@loc@TMPa
			\fi
		\fi
	\else
		% load from list:
		\expandafter\pgfplotslistnew\expandafter\pgfplotstable@VARIANCE\expandafter{\pgfplotstable@variance@list}%
	\fi
	%
	\pgfplotslistnewempty\pgfplotstable@Xparsed
	%
	\pgfmathfloatcreate{0}{0.0}{0}%
	\let\pgfplotstable@S=\pgfmathresult
	\let\pgfplotstable@Sxx=\pgfmathresult
	\let\pgfplotstable@Sx=\pgfmathresult
	\let\pgfplotstable@Sy=\pgfmathresult
	\let\pgfplotstable@Sxy=\pgfmathresult
	\pgfutil@loop
	\pgfplotslistcheckempty\pgfplotstable@X
	\ifpgfplotslistempty
		\pgfplots@loop@CONTINUEfalse
	\else
		\pgfplots@loop@CONTINUEtrue
	\fi
	\ifpgfplots@loop@CONTINUE
		\pgfplotslistpopfront\pgfplotstable@X\to\pgfplotstable@x
		\pgfplotslistpopfront\pgfplotstable@Y\to\pgfplotstable@y
		%
		\pgfplotstableparsex{\pgfplotstable@x}%
		\let\pgfplotstable@x=\pgfmathresult
		\expandafter\pgfplotslistpushback\pgfmathresult\to\pgfplotstable@Xparsed
		\pgfplotstableparsey{\pgfplotstable@y}%
		\let\pgfplotstable@y=\pgfmathresult
		%
		\pgfplotslistcheckempty\pgfplotstable@VARIANCE
		\ifpgfplotslistempty
			\pgfmathfloatcreate{1}{1.0}{0}%
			\let\pgfplotstable@invsqr=\pgfmathresult
		\else
			\pgfplotslistpopfront\pgfplotstable@VARIANCE\to\pgfplotstable@variance
			\pgfmathfloatparsenumber{\pgfplotstable@variance}%
			\let\pgfplotstable@variance=\pgfmathresult
			\pgfmathfloatmultiply@{\pgfplotstable@variance}{\pgfplotstable@variance}%
			\let\pgfplotstable@sqr=\pgfmathresult
			\pgfmathfloatreciprocal@{\pgfplotstable@sqr}%
			\let\pgfplotstable@invsqr=\pgfmathresult
		\fi
		%
		\pgfmathfloatadd@{\pgfplotstable@S}{\pgfplotstable@invsqr}%
		\let\pgfplotstable@S=\pgfmathresult
		%
		\pgfmathfloatmultiply@{\pgfplotstable@x}{\pgfplotstable@invsqr}%
		\let\pgfplots@table@accum=\pgfmathresult
		\pgfmathfloatadd@{\pgfplotstable@Sx}{\pgfplots@table@accum}%
		\let\pgfplotstable@Sx=\pgfmathresult
		%
		\pgfmathfloatmultiply@{\pgfplotstable@x}{\pgfplots@table@accum}%
		\let\pgfplots@table@accum=\pgfmathresult
		\pgfmathfloatadd@{\pgfplotstable@Sxx}{\pgfplots@table@accum}%
		\let\pgfplotstable@Sxx=\pgfmathresult
		%
		\pgfmathfloatmultiply@{\pgfplotstable@y}{\pgfplotstable@invsqr}%
		\let\pgfplots@table@accum=\pgfmathresult
		\pgfmathfloatadd@{\pgfplotstable@Sy}{\pgfplots@table@accum}%
		\let\pgfplotstable@Sy=\pgfmathresult
		%
		\pgfmathfloatmultiply@{\pgfplotstable@x}{\pgfplots@table@accum}%
		\let\pgfplots@table@accum=\pgfmathresult
		\pgfmathfloatadd@{\pgfplotstable@Sxy}{\pgfplots@table@accum}%
		\let\pgfplotstable@Sxy=\pgfmathresult
	\pgfutil@repeat
	%
	\pgfmathparse{\pgfplotstable@S * \pgfplotstable@Sxx - \pgfplotstable@Sx *\pgfplotstable@Sx}%
	\let\pgfplotstable@delta=\pgfmathresult
	%
	\pgfmathparse{(\pgfplotstable@S * \pgfplotstable@Sxy - \pgfplotstable@Sx * \pgfplotstable@Sy) / \pgfplotstable@delta}%
	\let\pgfplotstable@a=\pgfmathresult
	%
	\pgfmathparse{(\pgfplotstable@Sxx * \pgfplotstable@Sy - \pgfplotstable@Sx * \pgfplotstable@Sxy) / \pgfplotstable@delta}%
	\let\pgfplotstable@b=\pgfmathresult
	%
	\pgfplotslistnewempty\pgfplotstable@RESULT
	\pgfplotslistforeachungrouped\pgfplotstable@Xparsed\as\pgfplotstable@x{%
		\pgfmathfloatmultiply@{\pgfplotstable@x}{\pgfplotstable@a}%
		\let\pgfplotstable@tmp=\pgfmathresult
		\pgfmathfloatadd@{\pgfplotstable@tmp}{\pgfplotstable@b}%
		\ifx\pgfplotstableparseylogbase\pgfutil@empty
		\else
			\pgfplotstableparseyinv@{\pgfmathresult}%
		\fi
		\pgfmathfloattosci{\pgfmathresult}%
		\expandafter\pgfplotslistpushback\pgfmathresult\to\pgfplotstable@RESULT
	}%
	\pgfmathfloattosci\pgfplotstable@a
	\let\pgfplotstable@a=\pgfmathresult
	%
	\pgfmathfloattosci\pgfplotstable@b
	\let\pgfplotstable@b=\pgfmathresult
	%
	\global\let\pgfplotstableregressiona\pgfplotstable@a%
	\global\let\pgfplotstableregressionb\pgfplotstable@b%
	\let\pgfplotsretval=\pgfplotstable@RESULT
	\pgfmath@smuggleone\pgfplotsretval
	\endgroup
}%
\def\pgfplotstable@linear@regression@prepare@mode#1#2{%
	\expandafter\def\csname pgfplotstableparse#2\endcsname##1{\pgfmathfloatparsenumber{##1}}%
	\expandafter\def\csname pgfplotstableparse#2logbase\endcsname{}%
	\edef\pgfplots@loc@TMPa{#1}%
	\def\pgfplots@loc@TMPb{auto}%
	\ifx\pgfplots@loc@TMPa\pgfplots@loc@TMPb
		\def\pgfplots@loc@TMPa{}% auto == empty string
	\fi
	%
	\ifx\pgfplots@loc@TMPa\pgfutil@empty
		% `auto' mode.
		\pgfplotsifinaxis{%
			\pgfplots@if{pgfplots@#2islinear}{%
				\def\pgfplots@loc@TMPa{linear}%
			}{%
				\def\pgfplots@loc@TMPa{log}%
			}%
		}{%
			\def\pgfplots@loc@TMPa{linear}%
		}%
	\fi
	%
	\def\pgfplots@loc@TMPb{linear}%
	\ifx\pgfplots@loc@TMPa\pgfplots@loc@TMPb
	\else
		\def\pgfplots@loc@TMPb{log}%
		\ifx\pgfplots@loc@TMPa\pgfplots@loc@TMPb
			\pgfplotsmathdefinemacrolnbase{pgfplotstableparse#2}{\pgfkeysvalueof{/pgfplots/log basis #2}}%
		\else
			\pgfplotsthrow{invalid argument}{\pgfplots@loc@TMPa}{Sorry, the value '#1' is unexpected for 'linear regression/#2mode'}\pgfeov%
		\fi
	\fi
}%


% Sorts table #3 according to options defined in #1 and writes result
% to #2.
%
% \pgfplotstablesort[sort key=<colname>]\result{<input table>}
% \pgfplotstablesort[sort key=<colname>,sort key source=<\table>]\result{<input table>}
\def\pgfplotstablesort{\pgfutil@ifnextchar[{\pgfplotstablesort@opt}{\pgfplotstablesort@opt[]}}
\def\pgfplotstablesort@opt[#1]#2{%
	\begingroup
	\pgfplotstableset{#1}%
	\pgfplotstablecollectoneargwithpreparecatcodes{%
		\pgfplotstablesort@{#2}%
	}%
}
\def\pgfplotstablesort@#1#2{%
	\pgfkeysgetvalue{/pgfplots/table/sort key}\pgfplotstable@sort@key@colname
	\pgfkeysgetvalue{/pgfplots/table/sort key from}\pgfplotstable@sort@key@table
	\pgfkeysgetvalue{/pgfplots/table/sort cmp}\pgfplotstable@sort@cmp
	%
	%
	\pgfplotstable@isloadedtable{#2}{%
		\pgfplotstablegetrowsof{#2}%
		\let\pgfplotstable@numrows=\pgfplotsretval
		%
		\pgfplotstabletranspose[input colnames to=,colnames from=]\pgfp@tmp{#2}%
		\def\pgfplotstable@table{#2}%
		\let\pgfplotstable@input@colnames=#2\relax
	}{%
		\pgfplotstableread{#2}\pgfp@tmp@in
		\pgfplotstablegetrowsof\pgfp@tmp@in
		\let\pgfplotstable@numrows=\pgfplotsretval
		\def\pgfplotstable@table{\pgfp@tmp@in}%
		\let\pgfplotstable@input@colnames=\pgfp@tmp@in
		%
		\pgfplotstabletranspose[input colnames to=,colnames from=]\pgfp@tmp\pgfp@tmp@in
	}%
	%
	\ifx\pgfplotstable@sort@key@colname\pgfutil@empty
	\else
		\ifx\pgfplotstable@sort@key@table\pgfutil@empty
			\t@pgfplots@toka=\expandafter{\pgfplotstable@table}%
			\t@pgfplots@tokb=\expandafter{\pgfplotstable@sort@key@colname}%
			\edef\pgfplots@loc@TMPa{{\the\t@pgfplots@tokb}\noexpand\of{\the\t@pgfplots@toka}}%
			\expandafter\pgfplotstablegetcolumn\pgfplots@loc@TMPa\to\pgfplotstable@sort@key
		\else
			\t@pgfplots@toka=\expandafter{\pgfplotstable@sort@key@colname}%
			\t@pgfplots@tokb=\expandafter{\pgfplotstable@sort@key@table}%
			\edef\pgfplotstable@loc@TMPa{%
				\noexpand\pgfplotstablegetcolumn{\the\t@pgfplots@toka}\noexpand\of{\the\t@pgfplots@tokb}\noexpand\to\noexpand\pgfplotstable@sort@key}%
			\pgfplotstable@loc@TMPa
		\fi
	\fi
	\pgfplotsarraynewempty\pgfplotstable@sort@key@array
	\pgfplotslistforeachungrouped{\pgfplotstable@sort@key}\as\pgfplotstable@loc@TMPa{%
		\expandafter\pgfplotsarraypushback\expandafter{\pgfplotstable@loc@TMPa}\to\pgfplotstable@sort@key@array
	}%
	%
	%
	% use a side effect: columns are integers, they are stored as
	% '\pgfp@tmp@<index>'. Thus, column names and array indices are
	% essentially the same! That simplifies the operation and explains
	% why we transposed the table. 
	% We simply interprete the transposed table as array. We don't
	% even need to copy anything.
	\pgfplotsarrayresize{\pgfp@tmp}{\pgfplotstable@numrows}%
	% Oh, we also need the index to access the sort key. Well, then we
	% did not win anything.
	\pgfplotsarrayforeachungrouped\pgfp@tmp\as\pgfplotstable@loc@TMPa{%
		\t@pgfplots@toka=\expandafter{\pgfplotstable@loc@TMPa}%
		\edef\pgfplotstable@loc@TMPa{\pgfplotsarrayforeachindex:{\the\t@pgfplots@toka}}%
		\pgfplotsarrayletentry{\pgfplotsarrayforeachindex}\of\pgfp@tmp=\pgfplotstable@loc@TMPa
	}%
	% extract current row index and row content:
	\def\pgfplotstable@sort@extractindex##1:##2{%
		\def\pgfmathresult{##1}%
		\def\pgfplotstable@row{##2}%
	}%
	%
	\expandafter\pgfplotsset\expandafter{\pgfplotstable@sort@cmp}%
	\pgfkeysgetvalue{/pgfplots/iflessthan/.@cmd}\pgfplotstable@sort@cmp@routine
	\def\pgfplotstable@sort@iflt##1##2##3##4{\pgfplotstable@sort@cmp@routine{##1}{##2}{##3}{##4}\pgfeov}%
	%
	\pgfkeysdefargs{/pgfplots/iflessthan}{##1##2##3##4}{%
		\expandafter\pgfplotstable@sort@extractindex##1%
		\pgfplotsarrayselect{\pgfmathresult}\of\pgfplotstable@sort@key@array\to\pgfplotstable@arga
		\expandafter\pgfplotstable@sort@extractindex##2%
		\pgfplotsarrayselect{\pgfmathresult}\of\pgfplotstable@sort@key@array\to\pgfplotstable@argb
		\pgfplotstable@sort@iflt{\pgfplotstable@arga}{\pgfplotstable@argb}{##3}{##4}%
	}%
	\pgfkeysdef{/pgfplots/array/unscope pre}{%
		% remove the row index:
		\pgfplotsarrayforeachungrouped\pgfp@tmp\as\pgfplotstable@loc@TMPa{%
			\expandafter\pgfplotstable@sort@extractindex\pgfplotstable@loc@TMPa%
			\pgfplotsarrayletentry{\pgfplotsarrayforeachindex}\of\pgfp@tmp=\pgfplotstable@row
		}%
		% restore original form:
		\pgfplotstabletranspose[input colnames to=,colnames from=]\pgfp@tmp@result\pgfp@tmp
		% technical, transport result out of the current scope ...
		\pgfplotstable@copy@to@globalbuffers\pgfp@tmp@result{\pgfplotstablenameof\pgfp@tmp}%
		% ... and use this lowlevel thing to transport the original column names.
		\global\let\pgfplotstable@colnames@glob=\pgfplotstable@input@colnames
	}%
	\pgfkeysdef{/pgfplots/array/unscope post}{}%
	\pgfplotsarraysort{\pgfp@tmp}%
	\endgroup
	\pgfplotstable@copy@globalbuffers@to{#1}%
}%

\pgfutil@IfUndefined{pgfcalendardatetojulian}{%
	\def\pgfcalendardatetojulian#1#2{\pgfplots@error{Sorry, you need to use \string\usepackage{pgfcalendar} before using date specific methods}}%
}{}%
\endinput
